
\section{Expressions}
A BALL expression is evaluated in a predefined hierarchy and returns a
single value. An expression can return a value of any type. The
precedence of the operators when evaluating an expression is the same
as the order they are explained in this section, with highest
precedence first. The associativity of the operators will be explained
in each section. The order in which operands are evaluated is
undefined, even when the expressions have side effects.

\subsection{Atomic Expressions}
\begin{verbatim}
atom_expression : 
    STRING
    | IDENTIFIER
    | NUMBER
    | "(" expression ")"
    | list_initializer
    | "item"
;

list_initializer : "[" variable_list "]"
                 | type "[" "]"
                 ;
variable_list : expression
              | variable_list "," expression
              ;
\end{verbatim} 

An atomic expression is an identifier, constant, list initializer, or
an expression in parentheses. Identifiers will return the value of the
variable bound to the identifier. If there is no binding with respect
to that identifier, the compiler should display an error.

Anonymous lists are specified as a sequence of expressions (separated
by commas) inside square brackets. Each expression must evaluate to
the same type, and the type of the newly created list will be set to
that common type. The compiler should report an error if it cannot
derive a common type. If the new list is an empty list, having no
expressions means the new object cannot derive its type. Therefore,
the type needs to be explicitly written by the user.

{\tt item} is a special keyword used by a {\tt where} expression. It
acts as a variable holding the element of the list being filtered that
is currently being checked. It can only be written in the expression
argument of {\tt where}. The compiler should throw an error if {\tt
  item} is detected anywhere other than a {\tt where} expression. For
more information, see section \ref{ref_expr_where}.

\subsection{Primary Expressions}
\begin{verbatim}
primary_expression : atom_expression
                   | function_call
;
function_call : IDENTIFIER "(" ")"
              | IDENTIFIER "(" argument_list ")"
;
argument_list : expression
              | argument_list "," expression
;
\end{verbatim} 

A primary expression is either an atomic expression or a function
call. Arguments are expressions specified inside a list separated by
commas and delimited by parentheses.  Functions can only be referenced
by identifiers. BALL does not support storing functions inside
constructs such as lists, and thus functions can only be referenced
through their name that exists in the global scope. In addition,
function names cannot be overloaded.

\subsection{Postfix Expressions}
Postfix expressions are left-associative.
\begin{verbatim}
postfix_expression : primary_expression
                   | postfix_expression "'s" identifier
                   | postfix_expression "where" "(" expression ")"
                   | postfix_expression "++"
                   | postfix_expression "--"
                   ;
\end{verbatim}

\subsubsection{Attribute Calling}
The \texttt{'s} accessor operator acts much like the period
(\texttt{.}) character in Java. The left operand of the \texttt{'s}
evaluates to either a \texttt{team} or \texttt{player} object. The
right operand is an identifier that represents the name of a {\tt
  stat} or attribute associated with the value of the left
operand. For example:
\begin{verbatim}
    number teamwins = Orioles's W;
\end{verbatim}
Assuming \texttt{Orioles} is a loaded team, and since \texttt{W} is a
built-in stat, this program successfully compiles and stores the value
of the Orioles' wins into \texttt{teamwins}.

\subsubsection{\texttt{where} Expression}\label{ref_expr_where}
\texttt{where} is used to filter lists. The left side evaluates to a
list. The right side is a boolean expression. Each element of the left
side is tested with the boolean expression. If and only if the boolean
expression returns true on the element, the element will be inside the
filtered and returned list. To reference the element being checked
inside the boolean expression, use the \texttt{item} keyword. For
example:

\begin{verbatim}
    list of player 200hitters = Reds's PLAYERS 
        where (item's AVG > .200);
\end{verbatim}

The initial list on which the filtering will take place is the list
\texttt{Reds's PLAYERS}, assuming that \texttt{Reds} is a team that
has been loaded. The final, filtered list, will be stored in the list
\texttt{200hitters}, and will contain only the players from the first
list who fit the requirement of having an AVG over .200.

\subsubsection{Postfix Operators}
The postfix operators \texttt{++} and \texttt{----} behave identically with C and Java's increment/decrement operators, as long as the left-hand side is a name of a \texttt{number} variable. The variable's value will be returned before it is incremented/decremented. \texttt{++} increments the variable and \texttt{----} decrements the variable.

\subsection{Unary Expressions}
\begin{verbatim}
unary_expression : postfix_expression
                 | "++" unary_expression
                 | "--" unary_expression
                 | primary_expression "from" unary_expression
                 | "any" unary_expression
                 ;
\end{verbatim}
Unary expressions are evaluated right to left (right-associative).

\subsubsection{Prefix Operators}
The prefix operators \texttt{++} and \texttt{----} behave identically with C and Java's increment/decrement operators, as long as the left-hand side is a name of a \texttt{number} variable. The variable's value will be returned after it is incremented/decremented. \texttt{++} increments the variable and \texttt{----} decrements the variable.

\subsubsection{\texttt{from} Expression}
\texttt{from} is a matcher operation. The right operand is a
\texttt{list}, and the left operand is any object. The first element
of the right operand that "matches" the left operand will be
returned. If the elements are \texttt{player} objects, matching is
defined as an equivalent \texttt{name} or equivalent \texttt{player}
object. For everything else, matching is defined as being the same
object, equivalent to the \texttt{is} keyword. [See \ref{ref_expr_comp}.]

However, what should \texttt{from} return when it cannot match the
left hand side with any element in the right hand side? The return
value, when used as the argument for the builtin function
\texttt{isValid}, must return a \texttt{false} value. The exact value
itself is implementation dependent.

\subsubsection{\texttt{any} Expression}
The right operand of \texttt{any} is a \texttt{list}. The expression returns one random element from the \texttt{list}.

\subsection{Multiplicative Operations}\label{multiplication}
\begin{verbatim}
multiplication_expression : 
        unary_expression
        | multiplication_expression "*" unary_expression
        | multiplication_expression "/" unary_expression
        | multiplication_expression "%" unary_expression
        ;
\end{verbatim}

All multiplicative operators are left-associative. The order in which the operands are evaluated (left first or right first) implementation dependent. The operand expressions must have type \texttt{number}. The result of the operator \texttt{*} is the product of the two operands. If an implementation implicitly stores numbers as both integers and floating point numbers, both operands must be converted to floating point if one of the operands has a floating-point internal type.
The result of the operator \texttt{/} is the division between the two operands. Result of division by zero is implementation-defined. Rules on conversion of the two operand types are identical with the \texttt{*} operator.
The operator \texttt{\%} returns the modulo of the two numbers. Rules of the division operator apply here as well.

\subsection{Addition Operations}\label{ref_expr_addition}
\begin{verbatim}
addition_expression : 
        multiplication_expression
        | addition_expression "+" multiplication_expression
        | addition_expression "-" multiplication_expression
        ;
\end{verbatim}

Both addition and subtraction operators are left-associative. The order in which the operands are evaluated is implementation dependent. The operand expressions for \texttt{-} must evaluate to type \texttt{number}. The operand expressions for \texttt{+} must evaluate to and behaves as shown in table \ref{plusstuff}.
\begin{table}[htdp]
\begin{center}
\begin{tabular}{|c|c|p{7cm}|}
\hline
Left Type & Right Type & Result\\
\hline
\texttt{number} & \texttt{number} & The two \texttt{number}s are added together.\\
\texttt{list of T} & \texttt{list of T} & The right \texttt{list} is appended to the left, into a new \texttt{list}.\\
\texttt{string} & any object & The string representation of the right is concatenated to the left.\\
any object & \texttt{string} & The right is concatenated to the string representation of the left.\\
\hline
\end{tabular}
\end{center}
\caption{Operand Evaluation and Behavior for \texttt{+}.}\label{plusstuff}
\end{table}%



Rules for internal conversion for numeric operand values are identical to the multiplication operator in section \ref{multiplication}. 
\texttt{-} returns the subtraction between the two operands. Rules for internal conversion for the operand values are identical to the addition operator.

\subsection{Comparison Expressions}\label{ref_expr_comp}
\begin{verbatim}
comparison_expression : 
        addition_expression
        | comparison_expression "is" addition_expression
        | comparison_expression "isnot" addition_expression
        | comparison_expression ">" addition_expression
        | comparison_expression "<" addition_expression
        | comparison_expression ">=" addition_expression
        | comparison_expression "<=" addition_expression
;
\end{verbatim}

Evaluation between a series of comparison operations is done from left
to right (all comparison operators are left-associative).  Comparison
operators can only compare two values of the same type.  The equality
operator \texttt{is} has behavior detailed in table in table
\ref{isbehavior}. The inequality operator \texttt{isnot} returns true
when \texttt{is} returns false and vice versa. The rest of the
comparison operators only work on numbers.

\begin{table}[htdp]
\begin{center}
\begin{tabular}{|c|p{8cm}|}
\hline
Type & Returns true only if\\
\hline
\texttt{number} & The two numbers are equal\\
\texttt{string} & The strings are equal (case sensitive)\\
anything else & Both refer to the same object in memory\\
\hline
\end{tabular}
\end{center}
\caption{Behavior of \texttt{is}.}\label{isbehavior}
\end{table}

For operators other than {\tt is} or {\tt isnot}, both operands need
to be {\tt number}s. Their behavior is that of Java's {\tt <}, {\tt
  <=}, {\tt >} and {\tt >=} operators when applied on two
floating-point numbers.

\subsection{Logical Expressions}
\begin{verbatim}
logical_not_expression : 
        comparison_expression
        | "not" logical_not_expression
        ;
logical_and_expression : 
        logical_not_expression
        | logical_and_expression "and" logical_not_expression
        ;
logical_or_expression : 
        logical_and_expression
        | logical_or_expression "or" logical_and_expression
        ;
\end{verbatim}

Logical expressions are ordered, with decreasing precedence, as NOT, AND, and OR expressions. Each of these expressions take operands of type \texttt{number} and returns the truth value of the expression.


\texttt{not} takes a single expression and returns \texttt{false} if the original expression is \texttt{true} and \texttt{true} if the original expression is \texttt{false}.
\texttt{and} takes two expressions and returns \texttt{true} if both expressions return \texttt{true}, \texttt{false} otherwise. \texttt{or} takes two expressions and returns \texttt{false} if both expressions return \texttt{false}, \texttt{true} otherwise.


The logical \texttt{or} expression is the expression with the lowest precedence, and is made of terms of expressions with higher precedence. An \texttt{expression} proper is a logical \texttt{or} expression.

\begin{verbatim}
expression : logical_or_expression
           ;
\end{verbatim}
