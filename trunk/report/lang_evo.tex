\chapter{Language Evolution}

\section{Implementation Evolution}

At first, BALL was just a mixture of ideas. Early in the brainstorming
sessions, we determined our goal should be to facilitate the
simulation of Baseball games: we had identified a need. Baseball is
one of the most studied games in terms of statistics. A language that
effectively makes use of all of this available data is without any
doubt going to help predicting and simulating the outcome of a game or
a season. Baseball was without a doubt a sport of choice due to its
structure: bases can be assimilated to computational states, and the
playoff is a binary problem: it can be assimilated to a game in
between a batter and a pitcher from each team.

\subsection{The initial idea}
As we started deciding on how a BALL program will look like, we were
first inclined towards creating a scripting language in the style of
AWK: our language would be focused on heavy manipulation of data,
which can come in handy when dealing with a lot of
statistics. Following this idea, we envisioned a scripting language
that would provide as many shortcuts as possible to compute statistics
and apply them to players and teams. Nevertheless, some imminent
issues came to our minds: would this provide a good environment for a
novice programmer (most likely to be a baseball fan) to simulate a
game step by step? Would our language be too much like a simulation
program and not enough like a programming language?

\subsection{The new BALL}
Our final decision was to create a compiled language. This was
definitely a more novice-friendly alternative, and most certainly a
better way to allow programmers to define exactly how they would like
to use the data to simulate a Baseball game. Nevertheless, we thought
we should not completely forget the benefits of a scripting
language. Could we make a language somewhat like SQL? We needed some
quick and easy ways to extract players from a team, items from a list
and filter lists. Eventually, after much more planning and discussing
through the good code wiki, we came up with SQL-like functionality
such as the \texttt{from} keyword which matches and returns an item from a
list, \texttt{where (expression)} which returns a sub-list of items that
satisfy the expression provided and the \texttt{any} keyword which retrieves
a random item from a list. Keeping some of the benefits of scripting
languages, we devised a very important tool: stats. Stats would be
declared as a function of other, predefined or previously defined
stats. Programmers would be able to retrieve new stats from players or
teams in the same way as attributes, but the primitive stats
referenced by the newly defined stat would be retrieved from the
player or team in question. Lastly, we compared our language to others
such as C and Java and decided we could simplify it greatly. To
minimize programming errors, BALL is Baseball simulation specialized
and only provides the tools necessary for simulating baseball
games. We also introduced english-like keywords to help novice
programmers (who just want to simulate baseball games!) jump in as
soon as possible.

\subsection{Implementation process}
Our coding process was entirely driven by goals. Our goals consisted
of small test files to be compiled successfully. Our first goal was
the ``hello world'' program:

\begin{verbatim}
print "playball!";
\end{verbatim}

We started by setting up our parser (BYACC/J) and have it accept our
grammar. The next step was our lexer (JFLEX). We quickly added regular
expressions for our tokens and returned the \texttt{print} and
StringConstant tokens to the parser. This way, after creating the
print statement parse tree node, our language successfully printed
``playball!'';

From this point on, distributing the work was much simpler: Each of
the members of the team would go ahead and implement a tree node of
their choice. Each tree node would be responsible for generating its
own java equivalent. However, since all of these tiny classes need to
work together harmoniously, each parse tree node was a team effort and
most of the coding was done during our long meetings.it is very hard
to track down a specific class to less then three people: whenever
some major changes were made, such as having the compiler output to a
java file and pile it instead of simply outputting to standard output,
required attention and participation from the entire team.

We were committed to keep our language details identical from he
moment we first wrote our reference manual. Examples of challenging
tasks that had to be completed:

Both stats (portable functions) and attributes (fields) would have to
be retrieved equally using the accessor \texttt{'s}.  the keywords
\texttt{any} and \texttt{from} would have to extract players from
teams as well as objects from lists.  \texttt{where} statements would
have to work on lists of any type.

We discussed all of these problems by reviewing the code on the big
screen and we were able to solve them all. For example, to make
\texttt{from} and \texttt{any} work on both lists and teams, we gave
teams three new attributes: BATTERS, PITCHERS and PLAYERS which can be
accessed using \texttt{team1's PLAYERS} and return
\texttt{list}s. Other challenges, such as \texttt{where} statements,
drove us to make major changes.


\section{Change Management process}

We used SVN on Google Code repository to track our changes. Each of us
became very familiar with all the functionality of SVN, each of us
branching to implement new functionality and merging to the trunk
whenever a task was completed and thoroughly tested.

We faced some challenges when merging together changes from different
team members, reverting and differentiating. We got used to it quickly
and made the most out of this version management system.
