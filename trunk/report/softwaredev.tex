\chapter{Development Environment}\label{DevelEnv}

\section{Software Development Environment}

The software development environment that was chosen to create the
BALL compiler was Eclipse, running on Ubuntu distribution of Linux
OS. Ubuntu was chosen as the OS because it is open source and was
easily available to all team members. Eclipse was chosen specifically
because of its seamless integration with both Java and Subversion
revision control system. Most of the other tools used were selected
because the majority of the team members were already familiar with
them. Google Code was specially helpful in bringing everything
together since it made available to us features that were invaluable
at every stage of our compiler design. For the majority of the early
stages of our project, the record of our brainstorming sessions were
kept in a wiki stored on Google Code. This was a very helpful resource
that made it easy to trace the language and feature set
evolution. Likewise, without the ability to store our version
controlled code on Google Code, the development cycle would have been
extremely difficult to coordinate. In summary the most important tools
used are listed in the section below.

Ubuntu Linux
Eclipse
Java
Google Code
Subversion
JFlex
BYACC/J

\section{Implementation Languages and Tools}

Java was used to develop and build our BALL compiler to leverage the
considerable expertise that our team members possessed. It was also
specially helpful that Java is highly portable and enabled our BALL
compiler to work wherever Java runs. Only Java standard libraries were
used in the creation of our compiler, to simplify and further increase
the portability of BALL. Jflex \& BYACC/J the Java flavor version of
Lex and Yacc were used to build our lexer and parser. Finally Ant was
used to write buildfiles to automate the compilation process of our
package, and a bash shell script is used to compile and run BALL
source files.  
