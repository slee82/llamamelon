\chapter{TestPlan}\label{TestPlan}

As a group we all felt a test driven development was crucial to bug
free code.  Test Driven Development in our case started with one line
simple grammars and simple programs.  We then would introduce small
additions to the grammar and write small programs to act as unit tests
for said additions to the grammar.  The following is a code snippet
from the unit testing program for arithmetic expressions:

\begin{singlespacing}
\begin{verbatim}

/*
 * Arithmetic Expr Test File
 */

number x, y, z;

x = 0;//x = 0
y = x + 1;//y=1
z = x / 10;//z=0

x = (z + 1) * 100; //x=100

y = z % 5 + 1;//y=1

1+2;//should get commented out in javacode
x+y;

function f1 () returns number:
    x = 13;
    return 15;
end

function f2() returns number:
    y = x * 5 % 6;
    return 2 + 4;
end

print f1() + 3;// 18
print ``f2() = `` + f2();//6
print ``x = `` + x;//13 BECAUSE ITS GLOBAL (WORLDLY)
print ``y = `` + y;//5

\end{verbatim}
\end{singlespacing}

In the code above, we extensively test arithmetic expressions.  For
instance, we check all operators on all types.  While not shown above,
testing was done intending to both successfully compile and break the
code. [A successful break is when our intended breakpoint broke and
  produced the intended error]  

More generally our development was very conducive to standard software
development phasing.  Initially, each member created his own testbed
where all new functionality underwent white-box and black-box testing.
Our number one rule for testing was ``never break the trunk.''  To
make sure the trunk was never broken when moving code from a sandbox
testing environment to the trunk, we always made sure at least two
people read the new code in full before importing.  Then after we
compiled the updated source code, integration testing was done.  By
doing this, we could ensure that all new components were fully
functional with all the already existing components.  Furthermore, we
could be sure that for the most part, no new bugs were introduced.
Unfortunately, some bugs did slip through to the trunk but they were
quickly squashed int the regression testing phase.  After the
previously mentioned tests were successfully completed and the
compiler was fully functional using the complete grammar, the
regression testing phase began.  Below is the code used for the first
phase of regression testing.  The code was built on initially defining
a print statement.  It then evolved to a more complex print statement.
After several more rounds of regression testing, the program evolved
to the code below.  In it, we test all the constructs outlined in the
reference manual.  It was in this phase that most of the smaller typos
and bugs were found and fixed.  For example there was a type in the
teamObj that referred to the attribute name and it should have been
upper case.  Furthermore, each step was checked to see that the
intended output was realized.  It is important to note that in this
program, accuracy was not our goal.  Accuracy testing was done only on
the more complex sim function outlined in the last part of the
language tutorial.  In that function, we noticed very accurate
predictions when testing over thousands of games.  The program output
generally similar results with small perturbation to the actual number
of wins and losses for each team.  

The programs used in the testing phases can be found in the Code
section.
