\chapter{Project Plan}

\section{Development Process}\label{Process}

In developing the BALL language, we followed the suggestion presented
by Prof. Aho during class: make a working prototype, and incrementally
add enhancements. During the early parts of the development, more
attention is focused on adding new features than testing, however each
team member must create his own test case with the changes he made.

Whenever a part of the language is finished, it is tested and reviewed
by the team. Feedback is then relayed to the team member responsible
for the subsystem (usually the one who merged the latest version to
trunk, as many modules are jointly developed by the team). There is
less of a focus on the traditional team roles, rather each team member
focuses on developing the compiler while also doing their job in team
administration.

\pagebreak
\section{Style Sheet}\label{Style}

\begin{singlespacing}
\begin{verbatim}
/*
 * COMS W4119 PROGRAMMING LANGUAGES AND TRANSLATORS FALL 2009
 * Team llamamelon - BALL language
 * StyleSheet.java - style sheet for Java classes
 */

package style;

// package declaration comes after header, separated by one line
/*
 * header contains course name, team name, language name, file name, and a short
 * description of what the file is.
 */

// imports come right after, separated by at least 1 empty line
import java.util.Arrays;

import java.io.*;
// imports can have empty lines separating them, especially when importing
// different trees

import javabackend.SimFunction;
import javabackend.TeamObj;

// single line comments can use // or /* (for more important comments)
// two-line comments written like this are also permitted. For 3 or more, use /*

/*
 * Very important single-line comments are made like this
 */

/*
 * Multi-line comments also use this format. The maximum width for each line in
 * a multi-line comment is 80 characters from the start of the line (not the
 * start of the comment, which is after the space after a '*'.
 * 
 * 1.  this is how to write lists with numbers.
 * 2.  When a list goes over a line, the next line starts under the start of the
 *     first line of the list.
 *     ...
 * 10. If the numbering goes over 10, entries 1 to 9 need to have 1 more space.
 *     This way all list text starts at the same place. Having a list of more 
 *     than 10 members is discouraged.
 */

/**
 * Comments before a class uses Javadoc style. It is REQUIRED, even if just one
 * sentence. Give a brief description of what role it has in the compiler. 
 * 
 * Class names are capitalized on each word. 
 * - playerObj - WRONG 
 * - PlayerObj - RIGHT
 */
public class StyleSheet extends StyleSuper implements StyleInterface {

    // comments follow the indentation of the non-commented statements
    
    // public constructor comes first.
    /**
     * Sets use to default value. 
     */
    public StyleSheet() {
        this.use = 1001;
    }

    /*
     * Comments describing a method (highly recommended for public methods, even
     * for just one line). Some details:
     *
     * 1. comments for private methods are recommended.
     * 2. Javadoc style is optional. Can both be /* or /** or even //
     * 3. comments for methods that override a superclass method / implement an
     *    interface method are highly recommended it's not already documented in
     *    the superclass / interface. Otherwise, it's optional.
     * 5. Methods that override builtin classes' methods like toString() or
     *    hashCode() don't need to be commented.
     * 
     * Method names
     * 1. first letter of name NOT capitalized. However, words after that are
     *    captialized.
     * 2. Order of methods:
     *    a. public
     *    b. protected
     *    c. <no scope>
     *    d. private
     *    e. public static
     *    ...
     *    h. private static
     *    the names themselves can go in any order.
     * 
     * 1. Annotations like @Override are OPTIONAL. They go after the comment.
     * 2. Public methods cannot have underscores.
     */
    public int sampleFun(int arg1, int arg2, Object arg3) {
        // no space here ^ between function name and opening parentheses.
        // single space between arguments, and btwn close paren and open brace

        int p = 0, q; // space after each variable name
        q = 10;
        
        // note the spaces on for loops
        for (int i = 0; i < 10, i++) {
            // inside of for loops are also indented
            p++;
            q += 3;
            if (q >= 20) // single-line ifs like this are permitted
                System.out.println("bleh");
        }
        
        /*
         * Inner variables don't have to be capitalized properly.
         * 
         * Anonymous classes have a space after parentheses and opening curly
         * brace. The inside is indented.
         */
        SimFunction mysim = new SimFunction() {
            public TeamObj doSim(TeamObj team1, TeamObj team2) {
                return team1;
            }
        };

        return 1037; // parentheses for return exprs aren't necessary
    }
    
    /**
     * Since toString() is already in object, this is technically unnecessary
     */
    @Override
    public String toString() {
        return "style sheet";
    }
    
    /*
     * Order of fields is just like order of methods. "final" goes after
     * non-final but inside the same bracket above. So, "protected final" goes
     * after "public final" or "protected" but before "private".
     * 
     * protected x
     * public static final y
     */
    public int use;
    
    public final String bleh = "bleh";
    
    private float[] arrayExample;
    // variables/fields are named like methods

}
\end{verbatim}
\end{singlespacing}

\pagebreak

\section{Timeline \& Logs}\label{TimelineLogs}
\subsection{Timeline of BALL Development}

\begin{center}
\begin{longtable}{|c|p{10cm}|}
\hline
Time & Event\\
\hline
9/18/09, 11:31a & Team Llamamelon Born.\\
9/21/09, 5:15p & First Team Llamamelon Meeting; Team Llamamelon
named. Sam pioneers the language purpose: the simulation of baseball
games.\\
9/21/09, 8:02p & Team Llamamelon announced to PLT Teaching Staff.\\
10/2/09, 4:42p & Alfred V. Aho tentatively approves BALL programming
language.\\
10/5/09, 1:04a & Buzzwords for initial Whitepaper finalized.\\
10/7/07, 3:17p & Final Whitepaper completed.\\
10/23/09, 2:52p & Sample programs designed, BALL envisioned to be a
scripting language à la AWK.\\
10/27/09, 4:14p & Google Code Group Created. Jordan initializes
Subversion repository and trains team for first 'commit.'\\
10/27/09, 5:30p & First major Team Llamamelon Meeting; language
roadmap created.\\
10/29/09, 10:30a & First meeting with Sinan; BALL changes from
scripted language to compiled language, based on Java.\\
10/30/09, 1:27p & Nathan creates first example program; the new
language roadmap is born.\\
10/31/09, 10:42p & Cipta begins creating the grammar, laying the
groundwork for BALL.\\
11/3/09, 1:05p & Language Reference Manual and Tutorial begun.\\
11/3/09, 4:43p & Language Grammar completed, Daniel creates and
finalizes complex simulation to show off the powerful grammar.\\
11/4/09, 10:26a & Final Language Reference Manual and Tutorial
completed.\\
11/15/09, 3:00p & First semi-weekly meeting. Full grammar implemented
using JFlex/BYACC-J. Lexer created a DFA!\\
11/18/09, 8:00p & Second semi-weekly meeting. Play.ball designed,
intermediate code generation begins.\\
11/19/09, 11:00a & Second meeting with Sinan; Language Reference
Manual and Tutorial approved; BALL is in full motion!\\
11/27/09, 12:43a & Play.Ball compiles and runs: ``Play Ball!'' [Hello
  world equivalent.]\\
12/3/09, 3:45p & Team Loader created; teams can now load into the
PlayerObj and TeamObj constructs.\\
12/5/09, 11:46a & Major all-day meeting; BALL presentation created and
beautified.\\
12/7/09, 4:30p & BALL presented to PLT. There was much rejoicing.\\
12/10/09, 4:37a & SymbolTable embraced as major component of BALL.\\
12/10/09, 3:02a & Simfunc.ball compiles and runs: simfunctions and
activate work.\\
12/12/09, 10:00a & Major all-day meeting; Much of language completed
during this time.\\
12/12/09, 10:48a & TeamStat and PlayerStat methods added: stats work
throughout BALL.\\
12/12/09, 7:59p & Lists become typed, major change in the language.\\
12/12/09, 8:55p & Logical, Boolean, Comparison expressions completed.\\
12/15/09, 12:56a & \texttt{where}, \texttt{from}, and \texttt{any} keywords working.\\
12/16/09, 2:02a & \texttt{'s} keyword works for all attributes, stats, and
lists inherent in team and player objects.\\
12/16/09, 6:08p & Unary operators completed.\\
12/17/09, 6:27p & Realized the class number is W4115, not
W4119. [Seriously.]\\
12/20/09, 10:30a & Final Major all-day meeting: everything completed.\\
12/20/09, 8:04p & Everything in the language works. Final presentation
compilation begun.\\
Now & Final presentation completed.\\

\hline
\caption{Token Example}
\label{token example}
\end{longtable}
\end{center}

\subsection{Project Log}

\begin{center}
\begin{supertabular}{|c|c|c|c|}

\hline
r540 & daniellasry & 2009-12-20 04:58:48 -0500 (Sun, 20 Dec 2009) & 2 lines\\
\multicolumn{4}{|l|}{cleaned up unit testing} \\
\hline
r539 & jordan.schau & 2009-12-20 04:43:57 -0500 (Sun, 20 Dec 2009) & 1 line \\
\multicolumn{4}{|p{8cm}|}{its late and i forgot 2 ;'s.  its late} \\
\hline
r538 & jordan.schau & 2009-12-20 04:43:18 -0500 (Sun, 20 Dec 2009) & 1 line \\
\multicolumn{4}{|p{8cm}|}{regression test = done.  all hundo lines or so} \\
\hline
r537 & daniellasry & 2009-12-20 04:42:22 -0500 (Sun, 20 Dec 2009) & 3 lines \\
\multicolumn{4}{|p{8cm}|}{TO CHECK FOR NULL TYPE} \\
\multicolumn{4}{|p{8cm}|}{-Use the function isValid() :)} \\
\hline
r536 & jordan.schau & 2009-12-20 04:32:30 -0500 (Sun, 20 Dec 2009) & 3 lines \\
\multicolumn{4}{|p{8cm}|}{DONT FORGET THE END!!!! ALWAYS CHECK FOR THE END!!!!!!!!   } \\
\multicolumn{4}{|p{8cm}|}{cue "The End" by the doors.  Seems appropriate eh?} \\
\hline
r535 & jordan.schau & 2009-12-20 04:15:20 -0500 (Sun, 20 Dec 2009) & 1 line \\
\multicolumn{4}{|p{8cm}|}{broken test!  (as it should!)} \\
\hline
r534 & daniellasry & 2009-12-20 03:30:12 -0500 (Sun, 20 Dec 2009) & 2 lines \\
\multicolumn{4}{|p{8cm}|}{Fixed initializing empty lists} \\
\hline
r533 & jordan.schau & 2009-12-20 03:29:05 -0500 (Sun, 20 Dec 2009) & 1 line \\
\multicolumn{4}{|p{8cm}|}{Testing suite and bug fixes} \\
\hline
r532 & daniellasry & 2009-12-20 02:03:37 -0500 (Sun, 20 Dec 2009) & 2 lines \\
\multicolumn{4}{|p{8cm}|}{More error reporting} \\
\hline
r531 & daniellasry & 2009-12-20 01:41:48 -0500 (Sun, 20 Dec 2009) & 2 lines \\
\multicolumn{4}{|p{8cm}|}{Committing latest changes in error reporting} \\
\hline
r530 & jordan.schau & 2009-12-20 01:24:49 -0500 (Sun, 20 Dec 2009) & 1 line \\
\multicolumn{4}{|p{8cm}|}{Commitment.} \\
\hline
r529 & jordan.schau & 2009-12-20 01:24:37 -0500 (Sun, 20 Dec 2009) & 1 line \\
\multicolumn{4}{|p{8cm}|}{Almost done with unit testing.  fiiiiinally} \\
\hline
r528 & jordan.schau & 2009-12-20 01:14:25 -0500 (Sun, 20 Dec 2009) & 6 lines \\
\multicolumn{4}{|p{8cm}|}{Captains Log - Day 43} \\
\multicolumn{4}{|p{8cm}|}{Unit testing has been completed} \\
\multicolumn{4}{|p{8cm}|}{The water is running out and the men have scurvy.} \\
\multicolumn{4}{|p{8cm}|}{Yet, I am not too worried.  } \\
\multicolumn{4}{|p{8cm}|}{Game on.} \\
\hline
r527 & jordan.schau & 2009-12-20 01:05:44 -0500 (Sun, 20 Dec 2009) & 1 line \\
\multicolumn{4}{|p{8cm}|}{more tests complete- stay tuned for more} \\
\hline
r526 & jordan.schau & 2009-12-20 00:46:59 -0500 (Sun, 20 Dec 2009) & 1 line \\
\multicolumn{4}{|p{8cm}|}{Test complete.~} \\
\hline
r525 & jordan.schau & 2009-12-20 00:39:12 -0500 (Sun, 20 Dec 2009) & 1 line \\
\multicolumn{4}{|p{8cm}|}{TEST COMPLETE! (ROBOT VOICE)} \\
\hline
r524 & jordan.schau & 2009-12-20 00:29:30 -0500 (Sun, 20 Dec 2009) & 1 line \\
\multicolumn{4}{|p{8cm}|}{Testing - SQUASHED BUG THAT WAS TEAMNAME = DEATH!  (NOT KIDDING)} \\
\hline
r523 & slee82 & 2009-12-20 00:22:07 -0500 (Sun, 20 Dec 2009) & 1 line \\
\hline
r522 & nathanmayermiller & 2009-12-20 00:08:09 -0500 (Sun, 20 Dec 2009) & 1 line \\
\multicolumn{4}{|p{8cm}|}{Added four tester teams: Amazin' Mavens, Awful Waffles, Hittin' Kittens, Pitchin' Bitches} \\
\hline
r521 & slee82 & 2009-12-20 00:03:30 -0500 (Sun, 20 Dec 2009) & 1 line \\
\multicolumn{4}{|p{8cm}|}{BALL shell script rev2} \\
\hline
r520 & nathanmayermiller & 2009-12-19 23:50:15 -0500 (Sat, 19 Dec 2009) & 1 line \\
\multicolumn{4}{|p{8cm}|}{Fixed Astros, added Cards and Cubs} \\
\hline
r519 & jordan.schau & 2009-12-19 23:45:08 -0500 (Sat, 19 Dec 2009) & 1 line \\
\multicolumn{4}{|p{8cm}|}{American League EAST~~ BOOM BOOM } \\
\hline
r518 & jordan.schau & 2009-12-19 23:39:09 -0500 (Sat, 19 Dec 2009) & 1 line \\
\multicolumn{4}{|p{8cm}|}{OOOOOOH CAAAAAAA~~NAAAAA~~DAAAAAAAAAAAAAAAAAAAAAAAAAAAAAAA!!!!!!!} \\
\hline
r517 & jordan.schau & 2009-12-19 23:35:56 -0500 (Sat, 19 Dec 2009) & 1 line \\
\multicolumn{4}{|p{8cm}|}{rays (not the devil rays)} \\
\hline
r516 & jordan.schau & 2009-12-19 23:33:07 -0500 (Sat, 19 Dec 2009) & 1 line \\
\multicolumn{4}{|p{8cm}|}{red sox (socks?)} \\
\hline
r515 & slee82 & 2009-12-19 23:31:40 -0500 (Sat, 19 Dec 2009) & 1 line \\
\multicolumn{4}{|p{8cm}|}{BALL sh script... rev1 takes cmd line argument the .ball source file} \\
\hline
r514 & jordan.schau & 2009-12-19 23:29:55 -0500 (Sat, 19 Dec 2009) & 1 line \\
\multicolumn{4}{|p{8cm}|}{yankees \#1  (if you dont count the dodgers!)} \\
\hline
r513 & slee82 & 2009-12-19 23:22:29 -0500 (Sat, 19 Dec 2009) & 1 line \\
\multicolumn{4}{|p{8cm}|}{updated ball.lex \& ball.y, removed system.err messages}\\
\hline
r512 & jordan.schau & 2009-12-19 23:21:28 -0500 (Sat, 19 Dec 2009) & 1 line \\
\multicolumn{4}{|p{8cm}|}{dbacks hello!} \\
\hline
r511 & jordan.schau & 2009-12-19 23:17:01 -0500 (Sat, 19 Dec 2009) & 1 line \\
\multicolumn{4}{|p{8cm}|}{padres - vamos los doyers!} \\
\hline
r510 & nathanmayermiller & 2009-12-19 23:15:42 -0500 (Sat, 19 Dec 2009) & 1 line \\
\multicolumn{4}{|p{8cm}|}{Added: Braves, Marlins, Mets, Nats, Phils. BOOM. NL EAST COMPLETE.} \\
\hline
r509 & jordan.schau & 2009-12-19 23:10:55 -0500 (Sat, 19 Dec 2009) & 1 line \\
\multicolumn{4}{|p{8cm}|}{adding giants - more teams MORE FUN!} \\
\hline
r508 & daniellasry & 2009-12-19 23:04:34 -0500 (Sat, 19 Dec 2009) & 2 lines \\
\multicolumn{4}{|p{8cm}|}{EVERYTHING FINALLY WORKS} \\
\hline
r507 & ciphwn & 2009-12-19 22:43:14 -0500 (Sat, 19 Dec 2009) & 1 line \\
\multicolumn{4}{|p{8cm}|}{simnectar working (hopefully)} \\
\hline
r505 & ciphwn & 2009-12-19 22:15:11 -0500 (Sat, 19 Dec 2009) & 1 line \\
\multicolumn{4}{|p{8cm}|}{working simnectar} \\
\hline
r502 & daniellasry & 2009-12-19 21:48:48 -0500 (Sat, 19 Dec 2009) & 3 lines \\
\multicolumn{4}{|p{8cm}|}{Correct simNectar} \\
\hline
r501 & daniellasry & 2009-12-19 21:25:29 -0500 (Sat, 19 Dec 2009) & 5 lines \\
\multicolumn{4}{|p{8cm}|}{MERGING TO TRUNK! committing :} \\
\multicolumn{4}{|p{8cm}|}{- HAPPY EASTER} \\
\multicolumn{4}{|p{8cm}|}{- error reporting} \\
\hline
r499 & nathanmayermiller & 2009-12-19 19:57:26 -0500 (Sat, 19 Dec 2009) & 1 line \\
\multicolumn{4}{|p{8cm}|}{Changed team files.} \\
\hline
r498 & nathanmayermiller & 2009-12-19 19:52:57 -0500 (Sat, 19 Dec 2009) & 1 line \\
\multicolumn{4}{|p{8cm}|}{Added HR stat to pitchers. NOICE.} \\
\hline
r497 & ciphwn & 2009-12-19 19:47:51 -0500 (Sat, 19 Dec 2009) & 3 lines \\
\multicolumn{4}{|p{8cm}|}{1. Iteration stmt counting uses auto-generated ID} \\
\multicolumn{4}{|p{8cm}|}{2. added bottom... functions} \\
\multicolumn{4}{|p{8cm}|}{3. statlist getting returns list of PlayerObj not list of Object} \\
\hline
r496 & nathanmayermiller & 2009-12-19 19:25:13 -0500 (Sat, 19 Dec 2009) & 1 line \\
\multicolumn{4}{|p{8cm}|}{Fixed ApostrExpr for teamname attribute.} \\
\hline
r494 & nathanmayermiller & 2009-12-19 19:17:28 -0500 (Sat, 19 Dec 2009) & 1 line \\
\multicolumn{4}{|p{8cm}|}{Added teamname attribute.} \\
\hline
r493 & ciphwn & 2009-12-19 19:09:49 -0500 (Sat, 19 Dec 2009) & 1 line \\
\multicolumn{4}{|p{8cm}|}{comitting simNectar.ball, compiles in BALL but javac fails} \\
\hline
r492 & ciphwn & 2009-12-19 18:52:40 -0500 (Sat, 19 Dec 2009) & 1 line \\
\multicolumn{4}{|p{8cm}|}{fixed foreach so identifier doesn't have to be declared, but unique} \\
\hline
r491 & ciphwn & 2009-12-19 18:31:26 -0500 (Sat, 19 Dec 2009) & 2 lines \\
\multicolumn{4}{|p{8cm}|}{1. Adding Sam's loop statements code} \\
\multicolumn{4}{|p{8cm}|}{2. "self" -> "item"} \\
\hline
r490 & ciphwn & 2009-12-19 18:12:39 -0500 (Sat, 19 Dec 2009) & 2 lines \\
\multicolumn{4}{|p{8cm}|}{1. fixed negative numbers with unary minus} \\
\multicolumn{4}{|p{8cm}|}{2. prevented deletion of files with '-k' in Run.java} \\
\hline
r483 & ciphwn & 2009-12-19 17:36:13 -0500 (Sat, 19 Dec 2009) & 1 line \\
\multicolumn{4}{|p{8cm}|}{deleting the useless 'testing' package.} \\
\hline
r482 & ciphwn & 2009-12-19 17:29:13 -0500 (Sat, 19 Dec 2009) & 1 line \\
\multicolumn{4}{|p{8cm}|}{Implemented topPlayers, topTeams} \\
\hline
r477 & slee82 & 2009-12-19 15:08:34 -0500 (Sat, 19 Dec 2009) & 1 line \\
\multicolumn{4}{|p{8cm}|}{merged if stmts and loop stmts} \\
\hline
r476 & ciphwn & 2009-12-19 14:49:50 -0500 (Sat, 19 Dec 2009) & 1 line \\
\multicolumn{4}{|p{8cm}|}{added builtins for top, etc} \\
\hline
r475 & slee82 & 2009-12-19 14:41:14 -0500 (Sat, 19 Dec 2009) & 1 line \\
\multicolumn{4}{|p{8cm}|}{s.ball} \\
\hline
r474 & nathanmayermiller & 2009-12-19 13:00:31 -0500 (Sat, 19 Dec 2009) & 1 line \\
\multicolumn{4}{|p{8cm}|}{Fixed fixFloat} \\
\hline
r473 & nathanmayermiller & 2009-12-19 12:33:55 -0500 (Sat, 19 Dec 2009) & 1 line \\
\multicolumn{4}{|p{8cm}|}{Added BF to Program.java. Chicka Chicka Boom Boom - UNN} \\
\hline
r472 & slee82 & 2009-12-19 12:33:01 -0500 (Sat, 19 Dec 2009) & 1 line \\
\multicolumn{4}{|p{8cm}|}{fixFloat to PrintStmt.java} \\
\hline
r471 & slee82 & 2009-12-19 12:31:53 -0500 (Sat, 19 Dec 2009) & 1 line \\
\multicolumn{4}{|p{8cm}|}{fixFloat to string concats in ArithmeticExpr.java} \\
\hline
r470 & nathanmayermiller & 2009-12-19 12:31:39 -0500 (Sat, 19 Dec 2009) & 1 line \\
\multicolumn{4}{|p{8cm}|}{Fixed PlayerObj BF assignment/constructor problem. -UNN} \\
\hline
r469 & nathanmayermiller & 2009-12-19 12:30:13 -0500 (Sat, 19 Dec 2009) & 1 line \\
\multicolumn{4}{|p{8cm}|}{Fixed parenthesis.} \\
\hline
r468 & slee82 & 2009-12-19 12:29:03 -0500 (Sat, 19 Dec 2009) & 1 line \\
\multicolumn{4}{|p{8cm}|}{added fixFloat to Tools.java} \\
\hline
r467 & nathanmayermiller & 2009-12-19 12:28:07 -0500 (Sat, 19 Dec 2009) & 1 line \\
\multicolumn{4}{|p{8cm}|}{Team files updated to add BF stat. BOOYAKASHA} \\
\hline
r466 & slee82 & 2009-12-19 12:28:04 -0500 (Sat, 19 Dec 2009) & 1 line \\
\multicolumn{4}{|p{8cm}|}{added BF to Loader.java} \\
\hline
r465 & nathanmayermiller & 2009-12-19 12:22:16 -0500 (Sat, 19 Dec 2009) & 1 line \\
\multicolumn{4}{|p{8cm}|}{Batters faced tester.} \\
\hline
r464 & slee82 & 2009-12-19 12:17:41 -0500 (Sat, 19 Dec 2009) & 1 line \\
\multicolumn{4}{|p{8cm}|}{added BF to pitchers in Loader.java } \\
\hline
r463 & nathanmayermiller & 2009-12-19 12:15:23 -0500 (Sat, 19 Dec 2009) & 1 line \\
\multicolumn{4}{|p{8cm}|}{Fixed Program.java to have 2B and 3B instead of DBL and TPL.} \\
\hline
r462 & nathanmayermiller & 2009-12-19 12:13:06 -0500 (Sat, 19 Dec 2009) & 1 line \\
\multicolumn{4}{|p{8cm}|}{Semi-colon error. OOP! Goofy me!} \\
\hline
r461 & nathanmayermiller & 2009-12-19 12:10:34 -0500 (Sat, 19 Dec 2009) & 1 line \\
\multicolumn{4}{|p{8cm}|}{Added BF methods/attribute to PlayerObj.} \\
\hline
r458 & daniellasry & 2009-12-19 01:52:21 -0500 (Sat, 19 Dec 2009) & 5 lines \\
\multicolumn{4}{|p{8cm}|}{IMPORVED ERROR REPORTING!} \\
\multicolumn{4}{|p{8cm}|}{- The parser now tells you the line number and the last matched token!} \\
\multicolumn{4}{|p{8cm}|}{- type 'prit "hello"' and see for yourselves} \\
\hline
r457 & daniellasry & 2009-12-19 00:34:25 -0500 (Sat, 19 Dec 2009) & 2 lines \\
\multicolumn{4}{|p{8cm}|}{MERGED!} \\
\hline
r429 & jordan.schau & 2009-12-16 21:08:34 -0500 (Wed, 16 Dec 2009) & 1 line \\
\hline
r423 & ciphwn & 2009-12-16 05:02:13 -0500 (Wed, 16 Dec 2009) & 3 lines \\
\multicolumn{4}{|p{8cm}|}{Merging nathan\_1 to trunk.} \\
\multicolumn{4}{|p{8cm}|}{changes: 's working} \\
\hline
r414 & ciphwn & 2009-12-15 17:14:36 -0500 (Tue, 15 Dec 2009) & 2 lines \\
\multicolumn{4}{|p{8cm}|}{1. fixed auto-runner to delete children class files} \\
\multicolumn{4}{|p{8cm}|}{2. stat semicolon/access problem fixd} \\
\hline
r412 & ciphwn & 2009-12-15 14:17:16 -0500 (Tue, 15 Dec 2009) & 1 line \\
\multicolumn{4}{|p{8cm}|}{Added auto compiler/runner. needs tools.jar.} \\
\hline
r410 & ciphwn & 2009-12-15 04:27:01 -0500 (Tue, 15 Dec 2009) & 4 lines \\
\multicolumn{4}{|p{8cm}|}{Merged cipta\_7 to trunk.} \\
\multicolumn{4}{|p{8cm}|}{1. Style sheet} \\
\multicolumn{4}{|p{8cm}|}{2. 'any' and 'from'} \\
\multicolumn{4}{|p{8cm}|}{3. Tools.java where random, top is probably going to be put} \\
\hline
r402 & ciphwn & 2009-12-14 18:59:27 -0500 (Mon, 14 Dec 2009) & 8 lines \\
\multicolumn{4}{|p{8cm}|}{COMMITING CIPTA\_6 TO TRUNK} \\
\multicolumn{4}{|p{8cm}|}{CHANGES} \\
\multicolumn{4}{|p{8cm}|}{1. 'where' is working. To get 'where' working expressions can now insert statements before the statement it currently lives in to pre-evaluate stuff (this makes writing loops tricky, but doable)} \\
\multicolumn{4}{|p{8cm}|}{2. Symbol table makes random tokens differently now. Everything starts with "tok\_" and a random  hex number.} \\
\multicolumn{4}{|p{8cm}|}{3. statements must check whether they have accumulated any inserts because of the expressions inside them. This is done in the "Stmt" base class. Subclasses need not worry but they must implement the "stmtCode" function instead of "code" now.} \\
\hline
r388 & jordan.schau & 2009-12-12 21:19:44 -0500 (Sat, 12 Dec 2009) & 1 line \\
\multicolumn{4}{|p{8cm}|}{updated to include the bool in Type.java and LogicalExpr.java (but shhh bools are a secret to the end user)} \\
\hline
r387 & ciphwn & 2009-12-12 20:15:00 -0500 (Sat, 12 Dec 2009) & 1 line \\
\multicolumn{4}{|p{8cm}|}{merged changes from cipta\_6 to trunk} \\
\hline
r382 & ciphwn & 2009-12-12 16:43:47 -0500 (Sat, 12 Dec 2009) & 1 line \\
\multicolumn{4}{|p{8cm}|}{Merge list lintializer to trunk} \\
\hline
r379 & jordan.schau & 2009-12-12 16:31:25 -0500 (Sat, 12 Dec 2009) & 1 line \\
\hline
r378 & jordan.schau & 2009-12-12 15:45:45 -0500 (Sat, 12 Dec 2009) & 1 line \\
\multicolumn{4}{|p{8cm}|}{arithmetic expressions are working - butter} \\
\hline
r366 & ciphwn & 2009-12-12 13:52:05 -0500 (Sat, 12 Dec 2009) & 1 line \\
\multicolumn{4}{|p{8cm}|}{myfun2 syntax error} \\
\hline
r364 & nathanmayermiller & 2009-12-12 13:48:38 -0500 (Sat, 12 Dec 2009) & 1 line \\
\multicolumn{4}{|p{8cm}|}{PlayerStat methods; changed double ip to float ip.} \\
\hline
r363 & ciphwn & 2009-12-12 13:45:13 -0500 (Sat, 12 Dec 2009) & 1 line \\
\multicolumn{4}{|p{8cm}|}{added '()'} \\
\hline
r362 & nathanmayermiller & 2009-12-12 13:41:10 -0500 (Sat, 12 Dec 2009) & 1 line \\
\multicolumn{4}{|p{8cm}|}{Added TeamStat methods} \\
\hline
r361 & ciphwn & 2009-12-12 13:30:42 -0500 (Sat, 12 Dec 2009) & 4 lines \\
\multicolumn{4}{|p{8cm}|}{- merged stats with simfunction} \\
\multicolumn{4}{|p{8cm}|}{- added smart code indenting} \\
\multicolumn{4}{|p{8cm}|}{- a few builtin stats} \\
\multicolumn{4}{|p{8cm}|}{- original expr => AtomicExpr} \\
\hline
r355 & daniellasry & 2009-12-10 18:03:32 -0500 (Thu, 10 Dec 2009) & 2 lines \\
\multicolumn{4}{|p{8cm}|}{Committed my branch to the trunk. Check out simfunc.ball} \\
\hline
r341 & daniellasry & 2009-12-09 20:13:06 -0500 (Wed, 09 Dec 2009) & 2 lines \\
\multicolumn{4}{|p{8cm}|}{Tested sim() a little more, going to brach off soon and implement activation statements.} \\
\hline
r340 & daniellasry & 2009-12-09 19:45:23 -0500 (Wed, 09 Dec 2009) & 2 lines \\
\multicolumn{4}{|p{8cm}|}{sim() working} \\
\hline
r339 & daniellasry & 2009-12-09 19:05:23 -0500 (Wed, 09 Dec 2009) & 2 lines \\
\multicolumn{4}{|p{8cm}|}{Modified javabackend: added a sim function, fixed IP in playerObj} \\
\hline
r330 & ciphwn & 2009-12-09 17:12:07 -0500 (Wed, 09 Dec 2009) & 2 lines \\
\multicolumn{4}{|p{8cm}|}{1. Fixed shift/reduce conflict because of empty function body} \\
\multicolumn{4}{|p{8cm}|}{2. Added special declaration for functions with empty bodies} \\
\hline
r329 & daniellasry & 2009-12-09 17:08:46 -0500 (Wed, 09 Dec 2009) & 2 lines \\
\multicolumn{4}{|p{8cm}|}{added regexp decimal numbers} \\
\hline
r327 & daniellasry & 2009-12-09 15:45:16 -0500 (Wed, 09 Dec 2009) & 2 lines \\
\multicolumn{4}{|p{8cm}|}{Working on the'sim()' function} \\
\hline
r323 & daniellasry & 2009-12-09 05:56:11 -0500 (Wed, 09 Dec 2009) & 2 lines \\
\multicolumn{4}{|p{8cm}|}{Removing presentations to save space (it adds up every time we branch) I will add it to the downloads.} \\
\hline
r322 & daniellasry & 2009-12-09 05:52:18 -0500 (Wed, 09 Dec 2009) & 2 lines \\
\multicolumn{4}{|p{8cm}|}{Going to bed: MERGING WITH TRUNK} \\
\hline
r318 & ciphwn & 2009-12-09 02:31:11 -0500 (Wed, 09 Dec 2009) & 10 lines \\
\multicolumn{4}{|p{8cm}|}{Functions and variable declarations now completely follow how the ball source is declared. That is, this is illegal:} \\
\multicolumn{4}{|p{8cm}|}{function x() returns nothing:} \\
\multicolumn{4}{|p{8cm}|}{	y();} \\
\multicolumn{4}{|p{8cm}|}{end} \\
\multicolumn{4}{|p{8cm}|}{function y() returns nothing:} \\
\multicolumn{4}{|p{8cm}|}{end} \\
\multicolumn{4}{|p{8cm}|}{because x doesn't know of y yet.} \\
\hline
r317 & ciphwn & 2009-12-09 01:56:19 -0500 (Wed, 09 Dec 2009) & 6 lines \\
\multicolumn{4}{|p{8cm}|}{Commiting changes from cipta\_3 branch to main trunk.} \\
\multicolumn{4}{|p{8cm}|}{1. Function definition basics working} \\
\multicolumn{4}{|p{8cm}|}{2. Variable declarations, both in the top level and inside function blocks, work. Top level variables gets pushed outside the class, vars inside functions stay inside.} \\
\multicolumn{4}{|p{8cm}|}{3. Assignment basics work. Variables can now be redefined.} \\
\multicolumn{4}{|p{8cm}|}{4. Type checking basics. Since there are only a handful of types this isn't hard to do.} \\
\hline
r304 & daniellasry & 2009-12-07 15:30:20 -0500 (Mon, 07 Dec 2009) & 2 lines \\
\multicolumn{4}{|p{8cm}|}{Fixed something} \\
\hline
r303 & daniellasry & 2009-12-07 15:28:08 -0500 (Mon, 07 Dec 2009) & 2 lines \\
\multicolumn{4}{|p{8cm}|}{Variable declarations now happen outside 'main' (on top of main).} \\
\hline
r299 & daniellasry & 2009-12-07 13:27:20 -0500 (Mon, 07 Dec 2009) & 2 lines \\
\multicolumn{4}{|p{8cm}|}{commited pres} \\
\hline
r298 & daniellasry & 2009-12-05 20:40:23 -0500 (Sat, 05 Dec 2009) & 2 lines \\
\multicolumn{4}{|p{8cm}|}{Updated presentation} \\
\hline
r297 & daniellasry & 2009-12-05 17:28:05 -0500 (Sat, 05 Dec 2009) & 3 lines \\
\multicolumn{4}{|p{8cm}|}{updated presentation} \\
\hline
r296 & nathanmayermiller & 2009-12-05 15:39:10 -0500 (Sat, 05 Dec 2009) & 1 line \\
\multicolumn{4}{|p{8cm}|}{Added comments to play.ball to test it.} \\
\hline
r295 & nathanmayermiller & 2009-12-05 12:57:41 -0500 (Sat, 05 Dec 2009) & 1 line \\
\multicolumn{4}{|p{8cm}|}{Fixed IP values (.1,.2) for playerObj} \\
\hline
r294 & daniellasry & 2009-12-05 12:22:47 -0500 (Sat, 05 Dec 2009) & 2 lines \\
\multicolumn{4}{|p{8cm}|}{Added The presentation} \\
\hline
r293 & daniellasry & 2009-12-04 17:15:05 -0500 (Fri, 04 Dec 2009) & 2 lines \\
\multicolumn{4}{|p{8cm}|}{Added support for integers. I can't get the regexp for decimals to work yet.} \\
\hline
r292 & jordan.schau & 2009-12-04 11:20:06 -0500 (Fri, 04 Dec 2009) & 3 lines \\
\multicolumn{4}{|p{8cm}|}{adding Daniels load application from the email he sent out to test it (as the tester and validator - because thats my job)} \\
\hline
r291 & daniellasry & 2009-12-04 04:30:42 -0500 (Fri, 04 Dec 2009) & 2 lines \\
\multicolumn{4}{|p{8cm}|}{Goign to add support for numbers tomorrow} \\
\hline
r290 & daniellasry & 2009-12-04 04:02:36 -0500 (Fri, 04 Dec 2009) & 2 lines \\
\multicolumn{4}{|p{8cm}|}{In this revision, play.ball reads two teams and prints them!!!!} \\
\hline
r289 & daniellasry & 2009-12-04 02:29:22 -0500 (Fri, 04 Dec 2009) & 2 lines \\
\multicolumn{4}{|p{8cm}|}{cleaned up more} \\
\hline
r288 & daniellasry & 2009-12-04 02:26:30 -0500 (Fri, 04 Dec 2009) & 2 lines \\
\multicolumn{4}{|p{8cm}|}{cleaned junk} \\
\hline
r287 & daniellasry & 2009-12-04 02:24:16 -0500 (Fri, 04 Dec 2009) & 2 lines \\
\multicolumn{4}{|p{8cm}|}{Merged my branch into trunk} \\
\hline
r283 & daniellasry & 2009-12-03 15:11:33 -0500 (Thu, 03 Dec 2009) & 2 lines \\
\multicolumn{4}{|p{8cm}|}{Added the Loader class. Worth a look. Returns a team. Handles a lot of errors.} \\
\hline
r281 & jordan.schau & 2009-12-03 14:22:07 -0500 (Thu, 03 Dec 2009) & 1 line \\
\multicolumn{4}{|p{8cm}|}{trivial comment change to test eclipse commit } \\
\hline
r280 & jordan.schau & 2009-12-03 13:59:13 -0500 (Thu, 03 Dec 2009) & 2 lines \\
\multicolumn{4}{|p{8cm}|}{updated astros and dodgers to match the data on baseball-reference.  removed the .txt file extension} \\
\hline
r279 & jordan.schau & 2009-12-03 13:30:09 -0500 (Thu, 03 Dec 2009) & 4 lines \\
\multicolumn{4}{|p{8cm}|}{Added ref man and tut.  moved to docs folder.} \\
\multicolumn{4}{|p{8cm}|}{xo} \\
\hline
r278 & daniellasry & 2009-12-03 08:03:58 -0500 (Thu, 03 Dec 2009) & 2 lines \\
\multicolumn{4}{|p{8cm}|}{Realized that Innings Pitches are actually decimal numbers, not ints. Reverted to the old CSVs and modified the player class to accept doubles for innings pitched.} \\
\hline
r277 & daniellasry & 2009-12-03 07:55:49 -0500 (Thu, 03 Dec 2009) & 2 lines \\
\multicolumn{4}{|p{8cm}|}{Added toString methods to the teams and Players. The load function works, commenting it.} \\
\hline
r276 & daniellasry & 2009-12-03 07:07:18 -0500 (Thu, 03 Dec 2009) & 3 lines \\
\multicolumn{4}{|p{8cm}|}{Fixed the constants in playerObj to be public final static int (so they can be accessed and yet not modified).} \\
\multicolumn{4}{|p{8cm}|}{Removed some random decimal values in the innings pitched (???).} \\
\hline
r275 & daniellasry & 2009-12-03 02:28:05 -0500 (Thu, 03 Dec 2009) & 2 lines \\
\multicolumn{4}{|p{8cm}|}{Removed unnecessary "Team Stats" header which made parsing difficult. Fixed Expr and PrintStmt so that the java print statement appears on one line.} \\
\hline
r274 & slee82 & 2009-12-02 21:33:12 -0500 (Wed, 02 Dec 2009) & 3 lines \\
\multicolumn{4}{|p{8cm}|}{Holler at cha boy. What up. TeamObj and playerObj are done. Uppercase Team, Lowercase player. SORRY, I messed up.} \\
\hline
r273 & jordan.schau & 2009-12-02 21:12:14 -0500 (Wed, 02 Dec 2009) & 2 lines \\
\multicolumn{4}{|p{8cm}|}{added teams!} \\
\hline
r272 & ciphwn & 2009-12-02 20:49:11 -0500 (Wed, 02 Dec 2009) & 1 line \\
\multicolumn{4}{|p{8cm}|}{adding next target program} \\
\hline
r271 & ciphwn & 2009-11-27 00:32:41 -0500 (Fri, 27 Nov 2009) & 5 lines \\
\multicolumn{4}{|p{8cm}|}{\_\_Integrating changes from the cipta\_1 branch\_\_} \\
\multicolumn{4}{|p{8cm}|}{1. Can print a hello world program to stdout (redirect stdout to a file to get the java program)} \\
\multicolumn{4}{|p{8cm}|}{2. Uses packages} \\
\multicolumn{4}{|p{8cm}|}{3. Uses an Ant build file} \\
\hline
r260 & ciphwn & 2009-11-23 14:01:01 -0500 (Mon, 23 Nov 2009) & 4 lines \\
\multicolumn{4}{|p{8cm}|}{Merging changes from branch to trunk.} \\
\multicolumn{4}{|p{8cm}|}{- added more comments} \\
\multicolumn{4}{|p{8cm}|}{- added header} \\
\multicolumn{4}{|p{8cm}|}{- added declaration for a symbol table, though not implemented yet} \\
\hline
r247 & jordan.schau & 2009-11-18 22:17:38 -0500 (Wed, 18 Nov 2009) & 2 lines \\
\multicolumn{4}{|p{8cm}|}{Deleting yacc.macosx} \\
\hline
r246 & jordan.schau & 2009-11-18 21:50:45 -0500 (Wed, 18 Nov 2009) & 2 lines \\
\multicolumn{4}{|p{8cm}|}{removing versions} \\
\hline
r245 & ciphwn & 2009-11-18 21:14:04 -0500 (Wed, 18 Nov 2009) & 4 lines \\
\multicolumn{4}{|p{8cm}|}{Removed 'y.tab.c'} \\
\multicolumn{4}{|p{8cm}|}{added test program.} \\
\hline
r244 & ciphwn & 2009-11-18 21:11:03 -0500 (Wed, 18 Nov 2009) & 3 lines \\
\multicolumn{4}{|p{8cm}|}{- Adding clean target} \\
\multicolumn{4}{|p{8cm}|}{- changed "ball\_simple.y" to just "ball.y"} \\
\hline
r243 & jordan.schau & 2009-11-18 21:07:18 -0500 (Wed, 18 Nov 2009) & 2 lines \\
\multicolumn{4}{|p{8cm}|}{removing junk} \\
\hline
r242 & ciphwn & 2009-11-18 21:06:06 -0500 (Wed, 18 Nov 2009) & 5 lines \\
\multicolumn{4}{|p{8cm}|}{First working lex and yacc file} \\
\multicolumn{4}{|p{8cm}|}{- no syntax directed translation yet} \\
\multicolumn{4}{|p{8cm}|}{- successfully detected the "print" keyword} \\
\multicolumn{4}{|p{8cm}|}{- successfully parsed the "playball!" string} \\
\hline
r241 & jordan.schau & 2009-11-18 21:04:19 -0500 (Wed, 18 Nov 2009) & 2 lines \\
\multicolumn{4}{|p{8cm}|}{removing my test junk} \\
\hline
r240 & jordan.schau & 2009-11-18 21:03:45 -0500 (Wed, 18 Nov 2009) & 2 lines \\
\multicolumn{4}{|p{8cm}|}{test} \\
\hline
r239 & ciphwn & 2009-11-18 20:58:23 -0500 (Wed, 18 Nov 2009) & 2 lines \\
\multicolumn{4}{|p{8cm}|}{adding a makefile} \\
\hline
r238 & jordan.schau & 2009-11-18 20:58:21 -0500 (Wed, 18 Nov 2009) & 1 line \\
\multicolumn{4}{|p{8cm}|}{added secrets.  SH!!!!} \\
\hline
r237 & jordan.schau & 2009-11-18 20:55:04 -0500 (Wed, 18 Nov 2009) & 1 line \\
\multicolumn{4}{|p{8cm}|}{added whitepaper...  fingers crossed} \\
\hline
r236 & slee82 & 2009-11-18 20:41:13 -0500 (Wed, 18 Nov 2009) & 1 line \\
\multicolumn{4}{|p{8cm}|}{test} \\
\hline
r1 & (no author) & 2009-10-27 19:14:40 -0400 (Tue, 27 Oct 2009) & 1 line \\
\multicolumn{4}{|p{8cm}|}<{Initial directory structure.} \\
\hline

\end{supertabular}
\end{center}

