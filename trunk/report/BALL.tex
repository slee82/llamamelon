%% Based on a TeXnicCenter-Template by Tino Weinkauf.
%%%%%%%%%%%%%%%%%%%%%%%%%%%%%%%%%%%%%%%%%%%%%%%%%%%%%%%%%%%%%

%%%%%%%%%%%%%%%%%%%%%%%%%%%%%%%%%%%%%%%%%%%%%%%%%%%%%%%%%%%%%
%% HEADER
%%%%%%%%%%%%%%%%%%%%%%%%%%%%%%%%%%%%%%%%%%%%%%%%%%%%%%%%%%%%%
\documentclass[letterpaper,oneside,12pt, pdftex]{report}
% Alternative Options:
%	Paper Size: a4paper / a5paper / b5paper / letterpaper / legalpaper / executivepaper
% Duplex: oneside / twoside
% Base Font Size: 10pt / 11pt / 12pt


%% Language %%%%%%%%%%%%%%%%%%%%%%%%%%%%%%%%%%%%%%%%%%%%%%%%%
\usepackage[USenglish]{babel} %francais, polish, spanish, ...
\usepackage[T1]{fontenc}
\usepackage[ansinew]{inputenc}

 \usepackage{listings}
  \usepackage{courier}
 \lstset{
         basicstyle=\footnotesize\ttfamily, % Standardschrift
         numbers=left,               % Ort der Zeilennummern
         numberstyle=\tiny,          % Stil der Zeilennummern
         stepnumber=1,               % Abstand zwischen den Zeilennummern
         numbersep=5pt,              % Abstand der Nummern zum Text
         tabsize=2,                  % Groesse von Tabs
         extendedchars=true,         %
         breaklines=true,            % Zeilen werden Umgebrochen
         keywordstyle=\color{red},
                frame=b,         
 %        keywordstyle=[1]\textbf,    % Stil der Keywords
 %        keywordstyle=[2]\textbf,    %
 %        keywordstyle=[3]\textbf,    %
 %        keywordstyle=[4]\textbf,   \sqrt{\sqrt{}} %
         stringstyle=\color{white}\ttfamily, % Farbe der String
         showspaces=false,           % Leerzeichen anzeigen ?
         showtabs=false,             % Tabs anzeigen ?
         xleftmargin=17pt,
         framexleftmargin=17pt,
         framexrightmargin=5pt,
         framexbottommargin=4pt,
         %backgroundcolor=\color{lightgray},
         showstringspaces=false      % Leerzeichen in Strings anzeigen ?        
 }
 \lstloadlanguages{% Check Dokumentation for further languages ...
         %[Visual]Basic
         %Pascal
         %C
         %C++
         %XML
         %HTML
         Java
 }
    %\DeclareCaptionFont{blue}{\color{blue}} 

  %\captionsetup[lstlisting]{singlelinecheck=false, labelfont={blue}, textfont={blue}}
 
\usepackage{lmodern} %Type1-font for non-english texts and characters

%% Packages for Graphics & Figures %%%%%%%%%%%%%%%%%%%%%%%%%%
\usepackage{graphicx} %%For loading graphic files
%\usepackage{subfig} %%Subfigures inside a figure
%\usepackage{tikz} %%Generate vector graphics from within LaTeX

%% Please note:
%% Images can be included using \includegraphics{filename}
%% resp. using the dialog in the Insert menu.
%% 
%% The mode "LaTeX => PDF" allows the following formats:
%%   .jpg  .png  .pdf  .mps
%% 
%% The modes "LaTeX => DVI", "LaTeX => PS" und "LaTeX => PS => PDF"
%% allow the following formats:
%%   .eps  .ps  .bmp  .pict  .pntg


%% Math Packages %%%%%%%%%%%%%%%%%%%%%%%%%%%%%%%%%%%%%%%%%%%%
\usepackage{amsmath}
\usepackage{amsthm}
\usepackage{amsfonts}


%% Line Spacing %%%%%%%%%%%%%%%%%%%%%%%%%%%%%%%%%%%%%%%%%%%%%
\usepackage{setspace}
%\singlespacing        %% 1-spacing (default)
\onehalfspacing       %% 1,5-spacing
%\doublespacing        %% 2-spacing


%% Other Packages %%%%%%%%%%%%%%%%%%%%%%%%%%%%%%%%%%%%%%%%%%%
%\usepackage{a4wide} %%Smaller margins = more text per page.
\usepackage{fancyhdr} %%Fancy headings
%\usepackage{longtable} %%For tables, that exceed one page


%%%%%%%%%%
% LINK STUFF - NEEDS A FIXIN
%


%%%%%%%%%%%%%%%%%%%%%%%%%%%%%%%%%%%%%%%%%%%%%%%%%%%%%%%%%%%%%
%% Remarks
%%%%%%%%%%%%%%%%%%%%%%%%%%%%%%%%%%%%%%%%%%%%%%%%%%%%%%%%%%%%%
%
% TODO:
% 1. Edit the used packages and their options (see above).
% 2. If you want, add a BibTeX-File to the project
%    (e.g., 'literature.bib').
% 3. Happy TeXing!
%
%%%%%%%%%%%%%%%%%%%%%%%%%%%%%%%%%%%%%%%%%%%%%%%%%%%%%%%%%%%%%

%%%%%%%%%%%%%%%%%%%%%%%%%%%%%%%%%%%%%%%%%%%%%%%%%%%%%%%%%%%%%
%% Options / Modifications
%%%%%%%%%%%%%%%%%%%%%%%%%%%%%%%%%%%%%%%%%%%%%%%%%%%%%%%%%%%%%

\input{BallOption.tex} %You need a file 'options.tex' for this
%% ==> TeXnicCenter supplies some possible option files
%% ==> with its templates (File | New from Template...).



%%%%%%%%%%%%%%%%%%%%%%%%%%%%%%%%%%%%%%%%%%%%%%%%%%%%%%%%%%%%%
%% DOCUMENT
%%%%%%%%%%%%%%%%%%%%%%%%%%%%%%%%%%%%%%%%%%%%%%%%%%%%%%%%%%%%%
\begin{document}

\pagestyle{empty} %No headings for the first pages.


%% Title Page %%%%%%%%%%%%%%%%%%%%%%%%%%%%%%%%%%%%%%%%%%%%%%%
%% ==> Write your text here or include other files.

%% The simple version:
%\title{BALL: Basic Athletic Logic Language}
%\author{Team Llamamelon}
%\date{} %%If commented, the current date is used.
%\maketitle

%% The nice version:
\input{BallTitle.tex} %%You need a file 'titlepage.tex' for this.
%% ==> TeXnicCenter supplies a possible titlepage file
%% ==> with its templates (File | New from Template...).


%% Inhaltsverzeichnis %%%%%%%%%%%%%%%%%%%%%%%%%%%%%%%%%%%%%%%
\tableofcontents %Table of contents
\cleardoublepage %The first chapter should start on an odd page.

\pagestyle{fancy} %Now display headings: headings / fancy / ...



%% Chapters %%%%%%%%%%%%%%%%%%%%%%%%%%%%%%%%%%%%%%%%%%%%%%%%%
%% ==> Write your text here or include other files.

%\input{intro} %You need a file 'intro.tex' for this.


%%%%%%%%%%%%%%%%%%%%%%%%%%%%%%%%%%%%%%%%%%%%%%%%%%%%%%%%%%%%%
%% ==> Some hints are following:

\input{preamble.tex}

\input{tut.tex}


%%%%%%%%%%
% Lang Ref Man
%%%%%%%%%%
\chapter{Language Reference Manual}\label{manual}

\section{Preamble}
Herein is the language reference manual for the BALL programming language. As a language, BALL is designed for user simplicity and ease of programming. Thus, users can create simulations and manipulate statistics with ease by using the constructs presented below.  

\subsection{Objectives}
The main objective of BALL is to allow users to manipulate simulations of baseball games by creating their own simulation functions to interact with teams. BALL's language constructs are primarily centered around this goal, as well as minor `helper' objectives, such as allowing statistical data to be viewed and compared. Another objective of BALL is to make simulation programming simple and straightforward, allowing the user to accomplish the necessary tasks without having to worry about obscure imports and packages.

\subsection{Abstract Features}
\subsubsection{Types}\label{types}
BALL is a typed language, specifically with five types, \texttt{number},
\texttt{string}, \texttt{list}, \texttt{team}, and \texttt{player}. The
\texttt{number} and \texttt{string} types are simple constants. This is further
explained in the section \ref{keywords} below. The type \texttt{list} is a
collection of objects, which are instances of types. \texttt{list} is therefore
typed itself (see \ref{of}). Thus it is possible to have a \texttt{list} of
\texttt{team} objects, \texttt{player} objects, \texttt{string}s,
\texttt{number}s, or other \texttt{list} objects. \texttt{list} objects are
homogenous, meaning they cannot contain objects of multiple types. The
\texttt{player} type is an object that has attributes of \texttt{string} and
\texttt{number} types. These attributes include the name of the player, whether
the player is a pitcher or not, and the statistics of that player. The
\texttt{team} type is an object that has attributes of \texttt{string},
\texttt{number}, and \texttt{list} types. These attributes include the name of
the team, the team's wins and losses, and the list of every player on the team.

\subsubsection{Language Features}
BALL is a compiled language. Source code is stored in files ending with the \texttt{.ball} extension (for example: \texttt{hello.ball}), and is provided to the BALL compiler, which translates the program into intermediate code (using Java). The compiler then compiles this intermediate Java code into bytecode, which is then executed inside the Java Virtual Machine. Thus the BALL compilation environment must exist within a Java-ready environment. BALL, like many other common programming languages, such as Java upon which BALL is built, is an imperative programming language. It is similar in structure to Java, though numerous changes and innovations, explained later, have been made to facilitate simpler programming.

\section{Lexical Conventions} \label{lexical convention}
BALL has the following five kinds of tokens: identifiers, keywords, constants, operators, and other separators. For example, the following line of code: \texttt{player aroid = "Alex Rodriguez" from Yankees;} breaks down to the following tokens:


\begin{table}[htdp]
\begin{center}
\begin{tabular}{|l|l|}
\hline
Code & Token\\
\hline
\texttt{player} & keyword \\

\texttt{aroid} & identifier\\

\texttt{=} & operator\\

\texttt{"Alex Rodriguez"} & constant\\

\texttt{from} & keyword\\

\texttt{Yankees} & identifier\\

\texttt{;} & separator\\
\hline
\end{tabular}
\caption{Token Example}
\end{center}
\label{token example}
\end{table}
BALL is a strictly case sensitive language, also much like Java.

\subsection{Comments}
BALL comments follow the same format as Java. Single line comments begin with \texttt{//} and are terminated at the end of the line. Multiline comments begin \texttt{/*} and end with \texttt{*/}.
\subsection{Identifiers}
Identifiers are similar to in Java, with the exception that BALL identifiers may begin with a digit, provided that there is at least one alphabet character or underscore in the identifier. This represents a major improvement over Java and allows variable names such as \texttt{2B} representing doubles. That is, identifiers are members of the language defined by the following regular expression: 
\begin{verbatim}
        [a-zA-Z0-9_]*[a-zA-Z_][a-zA-Z0-9_]*
\end{verbatim}

\subsection{Keywords} \label{keywords}
The following are the keywords in the BALL language: \texttt{from}, \texttt{any}, \texttt{stat}, \texttt{where}, \texttt{item}, \texttt{foreach}, \texttt{in}, \texttt{do}, \texttt{stopdo}, \texttt{times}, \texttt{player}, \texttt{team}, \texttt{list}, \texttt{of},  \texttt{number}, \texttt{string}, \texttt{print}, \texttt{if}, \texttt{then}, \texttt{else}, \texttt{return}, \texttt{simfunction}, \texttt{activate}, \texttt{function}, \texttt{returns}, \texttt{nothing}, \texttt{is}, \texttt{end}.
\subsubsection{\texttt{from}}
The \texttt{from} keyword selects an element within a \texttt{list} object. 
\begin{verbatim}
        player aroid = "Alex Rodriguez" from Yankees's BATTERS;
\end{verbatim}

\subsubsection{\texttt{any}}
The \texttt{any} keyword selects a random element within a \texttt{list} object.
\begin{verbatim}
        player dback = any Diamondbacks's PLAYERS;
\end{verbatim}

\subsubsection{\texttt{stat}}
The \texttt{stat} keyword defines a new statistic function. 
\begin{verbatim}
        stat AVG = H / AB;
\end{verbatim}

\subsubsection{\texttt{where}}
The \texttt{where} keyword selects a sublist within a \texttt{list} object. 
\begin{verbatim}
        list of player 300hitters = Dodgers's PLAYERS where (item's AVG > .300);
\end{verbatim}

\subsubsection{\texttt{item}}
The \texttt{item} keyword refers to each element of a \texttt{list} filtered by the \texttt{where} keyword. 
\begin{verbatim}
        list 300hitters = Dodgers's PLAYERS where (item's AVG > .300);
\end{verbatim}

\subsubsection{\texttt{foreach}} \label{foreach}
The \texttt{foreach} keyword iterates through each element in a \texttt{list} object. 
\begin{verbatim}
        foreach p in Royals's PITCHERS: 
               print p's name;
        end
\end{verbatim}

\subsubsection{\texttt{in}}
The \texttt{in} keyword is integral to the \texttt{foreach} loop structure. See \ref{foreach} above.

\subsubsection{\texttt{do}} \label{do}
The \texttt{do} keyword creates an iterating loop, similar to \texttt{do}/\texttt{while} or \texttt{for} in Java.
\begin{verbatim}
        number n=0;
        do 5 times:
            n++;
            print "hello "+n;
        end
\end{verbatim}

\subsubsection{\texttt{stopdo}} \label{stopdo}
The \texttt{stopdo} keyword is similar to the \texttt{break} keyword in Java. It escapes a \texttt{do} loop before the completion of all iterations.
\begin{verbatim}
    do 5 times:
        player dback = any Diamondbacks's PLAYERS;
        print dback's name;
        if (dback's name is "Jon Garland") then:
            stopdo;
        end
    end
\end{verbatim}

\subsubsection{\texttt{times}}
The \texttt{times} keyword is integral to the \texttt{do} loop. See \ref{do} above.

\subsubsection{\texttt{player}}
The \texttt{player} keyword defines a new \texttt{player} object.
\begin{verbatim}
    player nat = any Nationals's PLAYERS;
\end{verbatim}

\subsubsection{\texttt{team}}
The \texttt{team} keyword defines a new \texttt{team} object.
\begin{verbatim}
    team champs = sim(Astros, Reds, 1);
\end{verbatim}

\subsubsection{\texttt{list}} \label{list}
The \texttt{list} keyword defines a new \texttt{list} object.
\begin{verbatim}
    list of player 300hitters = RedSox's PLAYERS where (item's AVG > .300);
\end{verbatim}

\subsubsection{\texttt{of}}\label{of}
The \texttt{of} keyword is integral to the \texttt{list} declaration. See \ref{list}


\subsubsection{\texttt{number}}
The \texttt{number} keyword defines a new \texttt{number} instance.
\begin{verbatim}
    player bigZ = "Carlos Zambrano" from Cubs's PITCHERS;
    number n = bigZ's ERA;
\end{verbatim}

\subsubsection{\texttt{string}}
The \texttt{string} keyword defines a new \texttt{string} instance. 
\begin{verbatim}
    string name = "Lance Berkman";
\end{verbatim}

\subsubsection{\texttt{print}}
The \texttt{print} keyword prints something to the standard output.
\begin{verbatim}
    player vlad = "Vladimir Guerrero" from Angels's BATTERS;
    print vlad's AVG;
\end{verbatim}

\subsubsection{\texttt{if}} \label{if}
The \texttt{if} keyword opens an \texttt{if} statement.
\begin{verbatim}
    player bro = any Twins's PLAYERS;
    print bro's name;
    if (bro's name is "Joe Mauer") then:
            print "What an average!";
    end
\end{verbatim}

\subsubsection{\texttt{then}}
The \texttt{then} keyword is integral to the \texttt{if} statement. See \ref{if} above.

\subsubsection{\texttt{else}}
The else keyword begins and alternate statement to a previous if statement.
\begin{verbatim}
    player roar = any Tigers's PITCHER;
    print roar's name;
    if (roar's name is "Justin Verlander") then:
        print "Wow, heck of an ERA!";
    else:
        print "You're not Verlander!";
    end
\end{verbatim}

\subsubsection{\texttt{return}}
The \texttt{return} keyword is used to return either the winning team in a simfunction (see \ref{simfunction} below) or a value required for a returning function (see \ref{function}).
\begin{verbatim}
    function simple(player p) returns string:
        return p's name;
    end
\end{verbatim}

\subsubsection{\texttt{simfunction}} \label{simfunction}
The \texttt{simfunction} keyword is used to create a new simulation function used to carry out the actual simulations of games. [Note that \texttt{team1} and \texttt{team2} are implicitly passed to the \texttt{simfunction} whenever \texttt{sim} is called. These are implicit keywords.]
\begin{verbatim}
    simfunction badSim is:
        if (team1's W > team2's W ) then:
            return team1;
        else:
            return team2;
        end
    end

\end{verbatim}

\subsubsection{\texttt{activate}}\label{activate}
The \texttt{activate keyword} is used to apply a \texttt{simfunction} for use by the function \texttt{sim}.
\begin{verbatim}
    activate badSim;
\end{verbatim} 

\subsubsection{\texttt{function}} \label{function}
The \texttt{function} keyword is used to define a new function.
\begin{verbatim}
    function simple(player p) returns string:
        return p's name;
    end
\end{verbatim}

\subsubsection{\texttt{returns}}\label{returns}
The \texttt{returns} keyword is used to define the return type of a function. See \ref{function} above.

\subsubsection{\texttt{nothing}}\label{nothing}
The \texttt{nothing} keyword replaces Java's \texttt{void} keyword. It is a return type for functions that don't return anything.
\begin{verbatim}
    function simple(player p) returns nothing:
        print p's avg;
    end
\end{verbatim}

\subsubsection{\texttt{is}}\label{is}
The \texttt{is} keyword is integral to the \texttt{simfunction} declaration (see \ref{simfunction}). 

\subsubsection{\texttt{end}}\label{end}
The \texttt{end} keyword is required at the end of every loop, if-statement, function, and \texttt{simfunction} declaration. See \ref{foreach}, \ref{do}, \ref{if}, \ref{simfunction}, \ref{function} above.

\subsection{Constants}
Much like Java, BALL constants consist of two major categories, numeric constants and string constants. These are both touched upon briefly in section \ref{types} "Types."
\subsubsection{Numeric Constants}
Numeric constants, because they apply to the \texttt{number} type, are treated as only one type. Specifically, they can contain integer or float values. [Integer values, such as a number of hits; float values such as a batting average.] The unary minus operation is also supported, so a numeric constant can be negative or positive. Numeric constants are members of the language defined by the following regular expression. 
\begin{verbatim}
    0 | [-]?[0-9]*[.]?[0-9]+
\end{verbatim}

\subsubsection{String Constants}
String constants are character sequences. They must begin and end with quotation marks. Note that BALL does not have a char type. All character sequences must be within double-quotation marks and are treated as string constants. All special escaped characters that Java supports are supported by BALL, including but not limited to the ones in Table \ref{escape}.
\begin{table}[htdp]
\begin{center}
\begin{tabular}{|c|c|c|c|}
\hline
\texttt{\textbackslash n} & \texttt{\textbackslash t} & \texttt{\textbackslash \textbackslash} & \texttt{\textbackslash "}\\
\hline
\end{tabular}
\caption{Some Acceptable Escape Characters}\label{escape}
\end{center}
\end{table}%

\subsection{Operators}
BALL contains six types of operators: assignment, logical, comparison, arithmetic, incrementor, and accessor. Generally, these operators are similar to in other standard programming languages such as Java, but some notable exceptions, shown below, are implemented in BALL.

\subsubsection{Assignment Operators}
Table \ref{assigntable} shows the assignment operators.
\begin{table}[htdp]
\begin{center}
\begin{tabular}{|c|c|c|c|c|c|}
\hline
\texttt{=} & \texttt{+=} & \texttt{-=} & \texttt{*=} & \texttt{/=} & \texttt{\%=}\\
\hline
\end{tabular}
\caption{BALL's Assignment Operators}\label{assigntable}
\end{center}
\end{table}%
They act just like in Java.

\subsubsection{Logical Operators}
Table \ref{logical operators} shows the logical operators and their Java equivalents.
\begin{table}[htdp]
\begin{center}
\begin{tabular}{|c|c|}
\hline
BALL & Java\\
\hline
\texttt{and} & \texttt{\&\&} \\
\texttt{or} & \texttt{\textbar \textbar} \\
\hline
\end{tabular}
\caption{BALL's Logical Operators}\label{logical operators}
\end{center}
\end{table}%

\subsubsection{Comparison Operators}
Table \ref{comparison operators} shows the logical operators and their Java equivalents. 
\begin{table}[htdp]
\begin{center}
\begin{tabular}{|c|c|}
\hline
BALL & Java\\
\hline
\texttt{is} & \texttt{==} \\
\texttt{isnot} & \texttt{!=} \\
\texttt{\textgreater} & \texttt{\textgreater} \\
\texttt{\textless} & \texttt{\textless} \\
\texttt{\textgreater =} & \texttt{\textgreater =}\\
\texttt{\textless =} & \texttt{\textless =}\\
\hline
\end{tabular}
\caption{BALL's Comparison Operators}\label{comparison operators}
\end{center}
\end{table}%

\subsubsection{Arithmetic Operators}
The following are the arithmetic operators: 
Table \ref{arithmetictable} shows the arithmetic operators. They act just like in Java. The unary minus also fits in this category.
\begin{table}[htdp]
\begin{center}
\begin{tabular}{|c|c|c|c|c|}
\hline
\texttt{+} & \texttt{-} & \texttt{*} & \texttt{/} & \texttt{\%}\\
\hline
\end{tabular}
\caption{BALL's Arithmetic Operators}\label{arithmetictable}
\end{center}
\end{table}%

\subsubsection{Incrementor Operators}
Table \ref{incremtable} shows the incrementor operators. These can be prefix or postfix. In either case, the behavior is equivalent to Java.
\begin{table}[htdp]
\begin{center}
\begin{tabular}{|c|c|}
\hline
\texttt{++} & \texttt{----}\\
\hline
\end{tabular}
\caption{BALL's Incrementor Operators}\label{incremtable}
\end{center}
\end{table}%

\subsubsection{Accessor}
Table \ref{accessortable} shows the one accessor operator. It acts like a period in Java, returning the value of an attribute of an object. Thus, \texttt{p's name} in BALL is the same as \texttt{p.name} in Java.
\begin{table}[htdp]
\begin{center}
\begin{tabular}{|c|}
\hline
\texttt{'s}\\
\hline
\end{tabular}
\caption{BALL's Incrementor Operators}\label{accessortable}
\end{center}
\end{table}%

\subsection{Separators}
The following are the separators in BALL: whitespace, comma, semicolon, colon, the keyword \texttt{end}, parentheses, and square brackets. Some of these are delimiters. Whitespace is always ignored, except in strings. Square brackets are used for defining lists.

\section{Expressions}
A BALL expression is evaluated in a predefined hierarchy and returns a single value. An expression can return a value of any type. The precedence of the operators when evaluating an expression is the same as the order they are explained in this section, with highest precedence first. The associativity of the operators will be explained in each section. The order in which operands are evaluated is undefined, even when the expressions have side effects.

\subsection{Atomic Expressions}
\begin{verbatim}
atom_expression : identifier
                | number
                | string
                | list_initializer
                | "nothing"
                | "(" expression ")" // completes the cycle
                ;
\end{verbatim} 

An atomic expression is an identifier, number constant, string
constant, list initializer, and an expression in
parentheses. Identifiers will return the value of the variable bound
to the identifier, if it is a properly declared variable of type
\texttt{number}, \texttt{list}, \texttt{string}, \texttt{team}, or
\texttt{player}.

\subsection{Primary Expressions}
\begin{verbatim}
primary_expression : atom_expression
                   | function_call
                   ;
\end{verbatim} 
A primary expression is either an atomic expression or a function
call. Function calls in BALL are similar to function calls in C, with
a few exceptions.  Arguments are specified inside a list separated by
commas and delimited by parentheses.  Functions can only be referenced
by identifiers. BALL does not support storing functions inside
constructs such as lists, and thus functions can only be referenced
through their name that exists in the global scope. In addition,
function names cannot be overloaded.

\subsection{Postfix Expressions}
Postfix expressions are left-associative.
\begin{verbatim}
postfix_expression : primary_expression
                   | postfix_expression "'s" identifier
                   | postfix_expression "where" "(" expression ")"
                   | postfix_expression "++"
                   | postfix_expression "--"
                   ;
\end{verbatim}

\subsubsection{Attribute Calling}
The \texttt{'s} accessor operator acts much like the period
(\texttt{.}) character in Java. The left operand of the \texttt{'s}
evaluates to either a \texttt{team} or \texttt{player} object. The
right operand is an identifier that represents the name of a stat or
attribute associated with the value of the left operand. For example:
\begin{verbatim}
    number teamwins = Orioles's W;
\end{verbatim}
Assuming \texttt{Orioles} is a loaded team, and since \texttt{W} is a
built-in stat, this program successfully compiles and stores the value
of the Orioles' wins into \texttt{teamwins}.

\subsubsection{\texttt{where} Expression}
\texttt{where} is used to filter lists. The left side evaluates to a
list. The right side is a boolean expression. Each element of the left
side is tested with the boolean expression. If and only if the boolean
expression returns true on the element, the element will be inside the
filtered and returned list. To reference the element being checked
inside the boolean expression, use the \texttt{item} keyword. For
example:

\begin{verbatim}
    list of player 200hitters = Reds's PLAYERS 
        where (item's AVG > .200);
\end{verbatim}

The initial list on which the filtering will take place is the list \texttt{Reds's PLAYERS}, assuming that \texttt{Reds} is a team that has been loaded. The final, filtered list, will be stored in the list \texttt{200hitters}, and will contain only the players from the first list who fit the requirement of having an AVG over .200.

\subsubsection{Postfix Operators}
The postfix operators \texttt{++} and \texttt{----} behave identically with C and Java's increment/decrement operators, as long as the left-hand side is a name of a \texttt{number} variable. The variable's value will be returned before it is incremented/decremented. \texttt{++} increments the variable and \texttt{----} decrements the variable.

\subsection{Unary Expressions}
\begin{verbatim}
unary_expression : postfix_expression
                 | "++" unary_expression
                 | "--" unary_expression
                 | primary_expression "from" unary_expression
                 | "any" unary_expression
                 ;
\end{verbatim}
Unary expressions are evaluated right to left (right-associative).

\subsubsection{Prefix Operators}
The prefix operators \texttt{++} and \texttt{----} behave identically with C and Java's increment/decrement operators, as long as the left-hand side is a name of a \texttt{number} variable. The variable's value will be returned after it is incremented/decremented. \texttt{++} increments the variable and \texttt{----} decrements the variable.

\subsubsection{\texttt{from} Expression}
\texttt{from} is a matcher operation. The right operand is a
\texttt{list}, and the left operand is any object. The first element
of the right operand that "matches" the left operand will be
returned. If the elements are \texttt{player} objects, matching is
defined as an equivalent \texttt{name} or equivalent \texttt{player}
object. For everything else, matching is defined as being the same
object, equivalent to the \texttt{is} keyword. [See \ref{is}.]

However, what should \texttt{from} return when it cannot match the
left hand side with any element in the right hand side? The return
value, when used as the argument for the builtin function
\texttt{isValid}, must return a \texttt{false} value. The exact value
itself is implementation dependent.

\subsubsection{\texttt{any} Expression}
The right operand of \texttt{any} is a \texttt{list}. The expression returns one random element from the \texttt{list}.

\subsection{Multiplicative Operations}\label{multiplication}
\begin{verbatim}
multiplication_expression : 
        unary_expression
        | multiplication_expression "*" unary_expression
        | multiplication_expression "/" unary_expression
        | multiplication_expression "%" unary_expression
        ;
\end{verbatim}

All multiplicative operators are left-associative. The order in which the operands are evaluated (left first or right first) implementation dependent. The operand expressions must have type \texttt{number}. The result of the operator \texttt{*} is the product of the two operands. If an implementation implicitly stores numbers as both integers and floating point numbers, both operands must be converted to floating point if one of the operands has a floating-point internal type.
The result of the operator \texttt{/} is the division between the two operands. Result of division by zero is implementation-defined. Rules on conversion of the two operand types are identical with the \texttt{*} operator.
The operator \texttt{\%} returns the modulo of the two numbers. Rules of the division operator apply here as well.

\subsection{Addition Operations}\label{addition}
\begin{verbatim}
addition_expression : 
        multiplication_expression
        | addition_expression "+" multiplication_expression
        | addition_expression "-" multiplication_expression
        ;
\end{verbatim}

Both addition and subtraction operators are left-associative. The order in which the operands are evaluated is implementation dependent. The operand expressions for \texttt{-} must evaluate to type \texttt{number}. The operand expressions for \texttt{+} must evaluate to and behaves as shown in table \ref{plusstuff}.
\begin{table}[htdp]
\begin{center}
\begin{tabular}{|c|c|p{7cm}|}
\hline
Left Type & Right Type & Result\\
\hline
\texttt{number} & \texttt{number} & The two \texttt{number}s are added together.\\
\texttt{list of T} & \texttt{list of T} & The right \texttt{list} is appended to the left, into a new \texttt{list}.\\
\texttt{string} & any object & The string representation of the right is concatenated to the left.\\
any object & \texttt{string} & The right is concatenated to the string representation of the left.\\
\hline
\end{tabular}
\end{center}
\caption{Operand Evaluation and Behavior for \texttt{+}.}\label{plusstuff}
\end{table}%



Rules for internal conversion for numeric operand values are identical to the multiplication operator in section \ref{multiplication}. 
\texttt{-} returns the subtraction between the two operands. Rules for internal conversion for the operand values are identical to the addition operator.

\subsection{Comparison Expressions}
\begin{verbatim}
comparison_expression : 
        addition_expression
        | comparison_expression "is" addition_expression
        | comparison_expression "isnot" addition_expression
        | comparison_expression ">" addition_expression
        | comparison_expression "<" addition_expression
        | comparison_expression ">=" addition_expression
        | comparison_expression "<=" addition_expression
\end{verbatim}

Evaluation between a series of comparison operations is done from left to right (all comparison operators are left-associative).  Comparison operators can only compare two values of the same type.
The equality operator \texttt{is} has behavior detailed in table in table \ref{isbehavior}. The inequality operator \texttt{isnot} returns true when \texttt{is} returns false and vice versa. The rest of the comparison operators only work on numbers.

\begin{table}[htdp]
\begin{center}
\begin{tabular}{|c|p{8cm}|}
\hline
Type & Returns true only if\\
\hline
\texttt{number} & The two numbers are equal\\
\texttt{string} & The strings are equal (case sensitive)\\
anything else & Both refer to the same object in memory\\
\hline
\end{tabular}
\end{center}
\caption{Behavior of \texttt{is}.}\label{isbehavior}
\end{table}

\subsection{Logical Expressions}
\begin{verbatim}
logical_not_expression : 
        comparison_expression
        | "not" logical_not_expression
        ;
logical_and_expression : 
        logical_not_expression
        | logical_and_expression "and" logical_not_expression
        ;
logical_or_expression : 
        logical_and_expression
        | logical_or_expression "or" logical_and_expression
        ;
\end{verbatim}

Logical expressions are ordered, with decreasing precedence, as NOT, AND, and OR expressions. Each of these expressions take operands of type \texttt{number} and returns the truth value of the expression.


\texttt{not} takes a single expression and returns \texttt{false} if the original expression is \texttt{true} and \texttt{true} if the original expression is \texttt{false}.
\texttt{and} takes two expressions and returns \texttt{true} if both expressions return \texttt{true}, \texttt{false} otherwise. \texttt{or} takes two expressions and returns \texttt{false} if both expressions return \texttt{false}, \texttt{true} otherwise.


The logical \texttt{or} expression is the expression with the lowest precedence, and is made of terms of expressions with higher precedence. An \texttt{expression} proper is a logical \texttt{or} expression.

\begin{verbatim}
expression : logical_or_expression
           ;
\end{verbatim}


\section{Declarations}\label{Declarations}
Declarations specify how BALL will set aside a new variable within the
scope of the declaration. Declarations have the form:

\begin{verbatim}
declaration : type variable_declarators ";"
            ;
variable_declarators : variable_declarator
                     | variable_declarators "," variable_declarator
                     ;
variable_declarator : identifier
                    | identifier "=" expression
                    ;
\end{verbatim}

\texttt{variable\_declarators} is a comma-separated parentheses
delimited list of declarators. Each declarator contains a name and
optionally an expression that initializes the variable's value. The
expression's type must match the type of the declaration statement.

\subsection{Type Specifiers}

The type specifiers in BALL are: 

\begin{verbatim}
type : "number" | "string"
     | "list"   | "team"
     | "player" | "nothing"
     | list_type
     ;

list_type : "list" "of" type
          ;
\end{verbatim}

List types are declared with the type of their contents.  The type
\texttt{nothing} is only used when specifying return values for
functions. For example:

\begin{verbatim}
function printHello (string helloString) returns nothing:
   print helloString;
end
\end{verbatim}

\subsection{Meaning of Declarations}

Declarations map an identifier to an expression. Since an expression
can produce any \texttt{atom\_expression} , this allows for
declarations such as \texttt{number someNumber=9;} and \texttt{string
  someString="Hello!";}. But expressions may produce other
combinations, and their return value will be mapped to the
identifier. In this example, \texttt{aTeam} will be mapped to the
return value of the load function: \texttt{team aTeam =
  load(myTeam.team);} Furthermore, identifiers could be mapped to the
return values of arithmetic expressions or list manipulations.

\subsection{Declaring a List}\label{ListDecl}

\begin{verbatim}
list_initializer : "[" variable_list "]"
                 | type "[" "]"
                 ;
variable_list : expression
              | variable_list "," expression
              ;
\end{verbatim}

Lists are declared as a list of expressions (separated by commas)
between square brackets. Each expression must evaluate to the same
type as the list's expected content type.

BALL lists determine their type by looking at the arguments. If the
new list is an empty list, no arguments means lists cannot know what
type it contains. Therefore, the type of object the list is going to
hold needs to be specified for empty lists.

\subsection{Declaring Players and Teams}

Players and teams cannot be created inside a program. They must be
loaded from a team file formatted in CSV.

\subsection{Declaring a Stat}

Stats are small, portable functions that encapsulate just a single
expression that acts on a \texttt{player} or \texttt{team}
object. Whether the stat acts on a player or a team depends what stat
it calls to derive its value. Both \texttt{player} and \texttt{team}
objects have predefined stats. User-defined stats act on players if
the stats they call all act on players. The same goes for team
stats. However, if both player and team stats are called by a
user-defined stat, the compiler should fail to generate the
code. Similarly, if there are no stats called by a user-defined stat,
it should also fail because the use of the stat cannot be derived.

\begin{table}[htdp]
\begin{center}
\begin{tabular}{|c|c|p{5cm}|}
\hline
Name & Type & Meaning\\
\hline
\texttt{IP} & Pitcher & Innings Pitched\\
\texttt{K} & Pitcher & Strikeouts\\
\texttt{H} & Pitcher & Hits Allowed\\
\texttt{BB} & Pitcher & Walks Allowed\\
\texttt{ER} & Pitcher & Earned Runs\\
\texttt{AB} & Batter & At Bats\\
\texttt{R} & Batter & Runs\\
\texttt{H} & Batter & Hits\\
\texttt{2B} & Batter & Doubles\\
\texttt{3B} & Batter & Triples\\
\texttt{HR} & Batter & Home Runs\\
\texttt{BB} & Batter & Walks\\
\hline
\end{tabular}
\end{center}
\caption{Primitive Stats for \texttt{player} objects}\label{playerstats}
\end{table}%

\begin{table}[htdp]
\begin{center}
\begin{tabular}{|c|p{5cm}|}
\hline
Name & Meaning\\
\hline
\texttt{W} & Games Won\\
\texttt{L} & Games Lost\\
\hline
\end{tabular}
\end{center}
\caption{Primitive Stats for \texttt{team} objects}\label{teamstats}
\end{table}%


\begin{table}[htdp]
\begin{center}
\begin{tabular}{|c|c|c|p{5cm}|}
\hline
Object Type & Name & Return Type & Meaning\\
\hline
\texttt{player} & \texttt{name} & \texttt{string} & Name loaded from CSV \\
\texttt{team} & \texttt{teamname} & \texttt{string} & Team name from
CSV \\
\texttt{team} & \texttt{BATTERS} & \texttt{list of player} & Players
marked as batters \\
\texttt{team} & \texttt{PITCHERS} & \texttt{list of player} & Players
marked as pitchers \\
\texttt{team} & \texttt{PLAYERS} & \texttt{list of player} & All
players \\
\hline
\end{tabular}
\end{center}
\caption{Builtin Attributes}\label{teamstats}
\end{table}%

When defining a new stat, the user will need to build upon primitive
stats so BALL can derive what object the stat is for. For example:

\begin{verbatim}
stat ERA = ER/IP*9;
\end{verbatim}

This line of code defines a new stat for pitchers: the Earned Run
Average. A user may now access this 
stat from any pitcher just like he would access a primitive stat:
\texttt{player1's ERA}.

Stats return numbers. The user should be able to collect a stat as
a number at any time with code such as: 

\begin{verbatim}
number player1ERA = player1's ERA;
\end{verbatim}

\section{Statements}\label{Stmts}
At the top leve, BALL is a sequence of statements. Although similar to
statements in C, BALL statements follow different rules. They fall
into three groups:

\begin{verbatim}
statement : function_definition
          | sim_function_definition
          | body_statement
          ;
body_statement : if_statement
               | iteration_statement
               | jump_statement
               | declaration
               | stat_declaration
               | expression_statement
               | activate_statement
               | print_statement
               | assignment_statement
               ;
\end{verbatim}

Body statements are statements that are allowed in the body of if
statements, functions, loops, and such. Function definitions must be
declared in the global scope. However, functions can only be called by
statements placed after they're defined. The same rule applies to
simfunctions.

\subsection{Function Definition}\label{FuncDef}
Function definitions are generated by the following grammar:

\begin{verbatim}
function_definition : 
        "function" identifier "(" parameter_list0 ")" "returns" type
                 ":" "end"
        "function" identifier "(" parameter_list0 ")" "returns" type
                 ":" body_statement_list "end"
                    ;
\end{verbatim}

Any statement starting with the \texttt{function} keyword is
immediately identified as a function definition. The return type of
the function is specified after the parameters, using the keyword
"returns". The function body can be empty. A function accepts a
comma-separated list of parameters specified by the grammar:

\begin{verbatim}
parameter_list0 :
                | parameter_list
                ;
parameter_list : parameter
               | parameter_list "," parameter
               ;
parameter : type identifier
          ;
\end{verbatim}

Parameter lists can be empty, meaning the function does not take any
arguments. An example function definition is as follows:

\begin{verbatim}
function iAddNumbers (number num1, number num2) returns number:
  return num1 + num2;
end
\end{verbatim}

\subsection{Simulation Function Definition}\label{SimDef}
Simulation function definitions require only a brief explanation since
they are in fact simpler to declare than regular functions:

\begin{verbatim}
sim_function_definition : "simfunction" identifier "is:"
        body_statement_list "end"
                        ;
\end{verbatim}

This grammar is very similar to function\_definition but differs in the
following ways: 

\begin{itemize}
\item
  simfunction definitions do not specify any parameters. The
  parameters are implicit: a simulation function only handles two
  \texttt{team} objects named \texttt{team1} and
  \texttt{team2}. \texttt{team1} and \texttt{team2} represent the two
  teams that the sim function is currently simulating for.
\item
  simfunction does not have an explicit return value. Again, this is
  because the return value is restricted to \texttt{team}.
\end{itemize}

\subsection{If Statements}

\begin{verbatim}
if_statement : 
        "if" "(" expression ")" "then:" body_statement_list "end"
        | "if" "(" expression ")" "then:" body_statement_list
                  else_statement "end"
        ;
\end{verbatim}

These are similar to Java if statements, but the delimiters are
\texttt{then:} and \texttt{end} instead of { and }.  expression should
be any boolean expression or other expression that resolves into a
boolean.

\subsection{Iteration Statements}

\begin{verbatim}
iteration_statement : "do:" body_statement_list "end"
                    | "do" expression "times:" body_statement_list
                            "end"
                    | "foreach" identifier "in" identifier ":"
                            body_statement_list "end"
                    ;
\end{verbatim}

The first and second lines produce do loops. A do loop simply loops a
specified amount of times. They 
come in two forms which can be illustrated by the following examples: 

\begin{verbatim}
number i = 1;
do:
  i *= 2;
  if (i >= 1024) then:
       stopdo;
   end
end
\end{verbatim}

This do loop is declared with \texttt{do:} and ends with
\texttt{end}. Everything in between will be repeated indefinitely or
until \texttt{stopdo;} is stated.

\begin{verbatim}
number i = 0;
do 5 times:
   i++
   print "looped" + i + "times!";
end
\end{verbatim}

This second do loop is performed exactly 5 times since that is the
number of loops specified. Note that the number of loops can be any
expression, and code such as \texttt{do num1+num2 times:} is
acceptable.  Finally, the third line of the grammar above produces
\texttt{foreach} statements. A \texttt{foreach} statement iterates
through a list and performs an action on each of the items in
sequence. A \texttt{foreach} statements may as well be halted with a
\texttt{stopdo;} statement. \texttt{foreach} loops are of the form:

\begin{verbatim}
print "The Mets roster is:";
foreach player in Mets's BATTERS:
   print player's name;
end
\end{verbatim}
 
 \subsection{Jump Statements}
The C Reference Manual describes jump statements with: `Jump
statements transfer control unconditionally.'' In BALL, we want to
prevent programmer mistakes by omitting some of such tools which many
programmers flag as ``bad practice.'' Hence, BALL only supports two
jump statements:

\begin{verbatim}
jump_statement : "stopdo" ";"
               | "return" expression ";"
               | "return" ";"
               ;
\end{verbatim}

A \texttt{stopdo;} statement may appear only in an iteration and
terminates the execution of the smallest enclosing such statement:
control passes to the statement following the newly terminated one.

\texttt{return} may only be used in functions. When invoked, the
function halts and returns to its caller. The function may return
nothing, as is the case with functions of return type \texttt{nothing}
(this also happens when letting the function flow off the end). Or the
function may return an expression with with type matching the
function's expected return type.

\subsection{Activate Statement}\label{activateStmt}
The activate statement is short, simple, and requires little explanation:

\begin{verbatim}
activate_statement : "activate" identifier ";"
                   ;
\end{verbatim}

\texttt{activate} is followed by an identifier, which should be the name of a
simulation function defined in the same BALL program. This statement
simply sets the simulation function as the \texttt{active} function for use
in the builtin function \texttt{sim}. For example:

\begin{verbatim}
activate mySimFunction;
sim(Astros, Mets, 3);
\end{verbatim}

This will return the winner of three games between the Astros and the
Mets using \texttt{mySimFunction} as simulation
function. \texttt{activate} may be called multiple times in the same
BALL program.

\subsection{Print Statement}

The print statement is very simple as well: 

\begin{verbatim}
print_statement : "print" expression ";"
                ;
\end{verbatim}

In a nutshell, \texttt{print} will print expression if expression
resolves to a string or the toString value of any team or player that
\texttt{expression} returns. It will also print any
numbers. Additionally, print accept concatenations of any players,
teams, strings or numbers. Example:

\begin{verbatim}
print player1's name + " is " + 5 + "th in the rankings!";
\end{verbatim}

\subsection{Assignment Statement}
The assignment statement is an opportunity to redefine the value of a
variable. The grammar for these assignments is:

\begin{verbatim}
assignment_statement : identifier assignment_operator expression ";"
                     ;
assignment_operator : "=" | "+=" | "-=" | "*=" | "/=" | "%=" ;
\end{verbatim}

A typical variable assignment will be of the form:

\begin{verbatim}
number x = 10;
x = 11; // final value is 11
\end{verbatim}

Assignments can be made with more than just the "=" operator. The
operators "+=", "-=", "*=", "/=", "\%=" are shortcuts for:

\begin{verbatim}
i += num1; // i = i + num1;
i -= num1; // i = i - num1;
i *= num1; // i = i * num1;
i /= num1; // i = i / num1;
i %= num1; // i = i % num1;
\end{verbatim}
 
\section{Scope}
All declarations made outside of functions, loops, or if-statements
are global. Functions themselves must also be global. However, the
scope of all declarations made inside of functions, loops, or if
statements ends at the end of the \texttt{body\_statement\_list}. The
end keyword marks the end of the deepest
\texttt{body\_statement\_list}.

\section{Built-in Functions}

\subsection{Load Function}
The \texttt{load} function is used to import external
comma-separated-value team files (see CSV Specification for csv format
and specification) into \texttt{team} objects in a file. The syntax
is:

\begin{verbatim}
load(string);
\end{verbatim}
Example: 
\begin{verbatim}
team Astros = load("Astros.team");
\end{verbatim}

\subsection{Rand Function}
The rand function is used to generate a random (float) number between
two given arguments. The syntax is:

\begin{verbatim}
rand(number, number);
\end{verbatim}
Example: 
\begin{verbatim}
number randomNum = rand(0,1);
\end{verbatim}

\subsection{Max Function}
The max function is used to determine the largest of two given
arguments. The syntax is:

\begin{verbatim}
max(number, number);
\end{verbatim}
Example: 
\begin{verbatim}
number larger = max(5,100); // 100
\end{verbatim}
 
\subsection{Min Function}
The min function is used to determine the smallest of two given arguments. The syntax is:

\begin{verbatim}
min(number, number);
\end{verbatim}
Example: 
\begin{verbatim}
number larger = min(5,100); // 5
\end{verbatim}

\subsection{Top Players Function}
The top function is used to generate a sublist of players representing
the top players in that list with respect to the stat. The syntax is:

\begin{verbatim}
topPlayers(number, list, player stat);
\end{verbatim}
Example: 
\begin{verbatim}
list of player top5b = topPlayers(5, Braves, AVG);
\end{verbatim}

\subsection{Top Teams Function}
The \texttt{topTeams} function is used to generate a sublist of teams
representing the top teams in that list with respect to the stat. The
list is sorted with teams having the biggest tested stat values
first. The syntax is:

\begin{verbatim}
topTeams(number, list, team stat);
\end{verbatim}
Example: 
\begin{verbatim}
list of team top5t = topTeams(5, [Braves, Dodgers, Orioles], W);
\end{verbatim}

\subsection{Bottom Functions}
The \texttt{bottom} functions are counterparts to the \texttt{top}
functions. Instead of descending sort, the resulting list is sorted
least stat value first.

\begin{verbatim}
list of player bot5a = bottomPlayers(5, Braves, AVG);
list of team bot5b = bottomTeams(5, [Braves, Dodgers, Orioles], W);
\end{verbatim}

\subsection{IsValid Function}
The \texttt{isValid} function is used to determine whether a
\texttt{from} expression returns a value successfully. When the
argument is what \texttt{from} returns in failure, \texttt{isValid}
returns \texttt{false}. Otherwise, it returns \texttt{true}.

\begin{verbatim}
isValid(any type);
\end{verbatim}
Example: 
\begin{verbatim}
isValid("a" from string[]); // false;
isValid(3 from number[]); // false;
isValid(-3 from [2, 4, 5, 1]); // false;
isValid(5 from [2, 4, 5, 1]); // true;
isValid(dodgers from [dodgers, astros, reds, yankees]); // true;
\end{verbatim}

\subsection{Sim Function}
The \texttt{sim} function is BALL's main function. When this function
is called, it takes the activated (see section \ref{activateStmt})
simfunction (see section \ref{simDef}) and runs it on the two teams a
number of times. The team that gets returned is the team that has the
larger number of wins. The syntax is:
\begin{verbatim}
sim(team,team,number);
\end{verbatim}
Example: 
\begin{verbatim}
sim(Rockies,Giants,5); 
\end{verbatim}

\section{CSV Specifications}\label{csv} 
The following is an example of a CSV used as a team file. This file
would be called Astros.team and contains information on the whole
team, batters, and pitchers, and their stats.

\begin{verbatim}
Team Name:Houston Astros
Header:W,L
74,88

Type:Batter
Header:Name,AB,R,H,2B,3B,HR,BB
Ivan Rodriguez,327,41,82,15,2,8,13
Lance Berkman,460,73,126,31,1,25,97
Kazuo Matsui,476,56,119,20,2,9,34
Miguel Tejada,635,83,199,46,1,14,19
Geoff Blum,381,34,94,14,1,10,33
Carlos Lee,610,65,183,35,1,26,41
Michael Bourn,606,97,173,27,12,3,63
Hunter Pence,585,76,165,26,5,25,58
Wandy Rodriguez,63,4,8,3,0,0,1
Roy Oswalt,49,4,6,0,0,0,2
Brian Moehler,42,0,1,0,0,0,1
Mike Hampton,37,6,12,1,0,1,2
Russ Ortiz,28,2,5,0,0,1,1
Bud Norris,16,1,3,0,0,0,0


Type:Pitcher
Header:Name,IP,H,ER,HR,BB,K,BF
Wandy Rodriguez,205.2,192,69,21,63,193,849
Roy Oswalt,181.1,183,83,19,42,138,757
Brian Moehler,154.2,187,94,21,51,91,694
Mike Hampton,112.0,128,66,13,46,74,494
Russ Ortiz,85.2,95,53,8,48,65,387
Bud Norris,55.2,59,28,9,25,54,249

\end{verbatim}

\section{Complete Grammar}\label{grammar}

\begin{singlespacing}
\begin{verbatim}
%token identifier
%token string
%token number

%%
/***PROGRAM***/
program : statement_list
        ;

statement_list : statement
               | statement_list statement
               ;

statement : function_definition
          | sim_function_definition
          | body_statement
          ;

body_statement_list : body_statement
                    | body_statement_list body_statement
                    ;

/*Body Statements are all statements except function declarations*/
body_statement : if_statement
               | iteration_statement
               | jump_statement
               | declaration
               | stat_declaration
               | expression_statement
               | activate_statement
               | print_statement
               | assignment_statement
               ;

function_definition : "function" identifier "(" parameter_list0 ")"
        "returns" type ":" body_statement_list "end"
                    ;

parameter_list0 :
                | parameter_list
                ;

parameter_list : parameter
               | parameter_list "," parameter
               ;

parameter : type identifier
          ;

sim_function_definition : "simfunction" identifier "is:"
        body_statement_list "end" 
                        ;

activate_statement : "activate" identifier ";"
                   ;

print_statement : "print" expression ";"
                ;

if_statement : 
        "if" "(" expression ")" "then:" body_statement_list "end"
        | "if" "(" expression ")" "then:" body_statement_list
                else_statement "end"
        ;

else_statement : "else:" body_statement_list
               ;

/**ITERATION_STATEMENT**/
iteration_statement : 
        "do:" body_statement_list "end"
        | "do" expression "times:" body_statement_list "end"
        | "foreach" identifier "in" identifier ":" body_statement_list
                "end"
        ;

/**JUMP_STATEMENT**/
jump_statement : "stopdo" ";"
               | "return" expression ";"
               | "return" ";"
               ;
              
/**ASSIGNMENT_STATEMENT**/ 
assignment_statement : identifier assignment_operator expression ";"
                     ;

assignment_operator : "=" | "+=" | "-=" | "*=" | "/=" | "%=" ;

/**DECLARATION**/
declaration : type variable_declarators ";"
            ;

variable_declarators : variable_declarator
                     | variable_declarators "," variable_declarator
                     ;

variable_declarator : identifier
                    | identifier "=" expression
                    ;

stat_declaration : "stat" identifier "=" stat_expression ";"
                 ;

stat_expression : stat_mult_expr
                | stat_expression "+" stat_mult_expr
                | stat_expression "-" stat_mult_expr
                ;

stat_mult_expr : stat_atom_expr
               | stat_mult_expr "*" stat_atom_expr
               | stat_mult_expr "/" stat_atom_expr
               | stat_mult_expr "%" stat_atom_expr
               ;

stat_atom_expr : identifier
               | number
               | "(" stat_expression ")"
               ;

/*TYPE*/
type : "number"
     | "string"
     | "list"
     | "team"
     | "player"
     | "nothing"
     | list_type
     ;

list_type : "list" "of" type
          ;


/**EXPRESSION_STATEMENT**/
expression_statement : ";"
                     | expression ";"
                     ;

/*EXPRESSION*/
expression : logical_or_expression
           ;

/*LOGICAL*/
logical_or_expression : 
        logical_and_expression
        | logical_or_expression "or" logical_and_expression
        ;

logical_and_expression : 
        logical_not_expression
        | logical_and_expression "and" logical_not_expression
        ;

logical_not_expression : 
        comparison_expression
        | "not" logical_not_expression
        ;

/*COMPARISON*/
comparison_expression : 
        addition_expression
        | comparison_expression "is"   addition_expression
        | comparison_expression "isnot" addition_expression
        | comparison_expression ">"    addition_expression
        | comparison_expression "<"    addition_expression
        | comparison_expression ">="   addition_expression
        | comparison_expression "<="   addition_expression
        ;

/*ARITHMETIC*/
addition_expression : 
        multiplication_expression
        | addition_expression "+" multiplication_expression
        | addition_expression "-" multiplication_expression
        ;

multiplication_expression : 
        unary_expression
        | multiplication_expression "*" unary_expression
                        | multiplication_expression "/" unary_expression
                          | multiplication_expression "%" unary_expression
                          ;

/*UNARY*/
unary_expression : postfix_expression
                 | "++" unary_expression
                 | "--" unary_expression
                 | primary_expression "from" unary_expression
                 | "any" unary_expression
                 ;

/*POSTFIX*/
postfix_expression : primary_expression
                   | postfix_expression "'s" identifier // attribute/stats call
                   | postfix_expression "where" "(" expression ")"
                   | postfix_expression "++"
                   | postfix_expression "--"
                   ;

/*PRIMARY*/
primary_expression : atom_expression
                   | function_call
                   ;

/*FUNCTION_CALL*/
function_call : identifier "(" ")"
              | identifier "(" argument_list ")"
              ;

argument_list : expression
              | argument_list "," expression
              ;

/*ATOM_EXPRESSION*/
atom_expression : identifier
                | number
                | string
                | list_initializer
                | "nothing"
                | "(" expression ")" // completes the cycle
                ;

list_initializer : "[" variable_list "]"
                 | "[" "]"
                 ;

variable_list : expression
              | variable_list "," expression
              ;
%%
\end{verbatim}
\end{singlespacing}


\chapter{Project Plan}

\section{Development Process}\label{Process}

In developing the BALL language, we followed the suggestion presented
by Prof. Aho during class: make a working prototype, and incrementally
add enhancements. During the early parts of the development, more
attention is focused on adding new features than testing, however each
team member must create his own test case with the changes he made.

Whenever a part of the language is finished, it is tested and reviewed
by the team. Feedback is then relayed to the team member responsible
for the subsystem (usually the one who merged the latest version to
trunk, as many modules are jointly developed by the team). There is
less of a focus on the traditional team roles, rather each team member
focuses on developing the compiler while also doing their job in team
administration.

\pagebreak
\section{Style Sheet}\label{Style}

\begin{singlespacing}
\begin{verbatim}
/*
 * COMS W4119 PROGRAMMING LANGUAGES AND TRANSLATORS FALL 2009
 * Team llamamelon - BALL language
 * StyleSheet.java - style sheet for Java classes
 */

package style;

// package declaration comes after header, separated by one line
/*
 * header contains course name, team name, language name, file name, and a short
 * description of what the file is.
 */

// imports come right after, separated by at least 1 empty line
import java.util.Arrays;

import java.io.*;
// imports can have empty lines separating them, especially when importing
// different trees

import javabackend.SimFunction;
import javabackend.TeamObj;

// single line comments can use // or /* (for more important comments)
// two-line comments written like this are also permitted. For 3 or more, use /*

/*
 * Very important single-line comments are made like this
 */

/*
 * Multi-line comments also use this format. The maximum width for each line in
 * a multi-line comment is 80 characters from the start of the line (not the
 * start of the comment, which is after the space after a '*'.
 * 
 * 1.  this is how to write lists with numbers.
 * 2.  When a list goes over a line, the next line starts under the start of the
 *     first line of the list.
 *     ...
 * 10. If the numbering goes over 10, entries 1 to 9 need to have 1 more space.
 *     This way all list text starts at the same place. Having a list of more 
 *     than 10 members is discouraged.
 */

/**
 * Comments before a class uses Javadoc style. It is REQUIRED, even if just one
 * sentence. Give a brief description of what role it has in the compiler. 
 * 
 * Class names are capitalized on each word. 
 * - playerObj - WRONG 
 * - PlayerObj - RIGHT
 */
public class StyleSheet extends StyleSuper implements StyleInterface {

    // comments follow the indentation of the non-commented statements
    
    // public constructor comes first.
    /**
     * Sets use to default value. 
     */
    public StyleSheet() {
        this.use = 1001;
    }

    /*
     * Comments describing a method (highly recommended for public methods, even
     * for just one line). Some details:
     *
     * 1. comments for private methods are recommended.
     * 2. Javadoc style is optional. Can both be /* or /** or even //
     * 3. comments for methods that override a superclass method / implement an
     *    interface method are highly recommended it's not already documented in
     *    the superclass / interface. Otherwise, it's optional.
     * 5. Methods that override builtin classes' methods like toString() or
     *    hashCode() don't need to be commented.
     * 
     * Method names
     * 1. first letter of name NOT capitalized. However, words after that are
     *    captialized.
     * 2. Order of methods:
     *    a. public
     *    b. protected
     *    c. <no scope>
     *    d. private
     *    e. public static
     *    ...
     *    h. private static
     *    the names themselves can go in any order.
     * 
     * 1. Annotations like @Override are OPTIONAL. They go after the comment.
     * 2. Public methods cannot have underscores.
     */
    public int sampleFun(int arg1, int arg2, Object arg3) {
        // no space here ^ between function name and opening parentheses.
        // single space between arguments, and btwn close paren and open brace

        int p = 0, q; // space after each variable name
        q = 10;
        
        // note the spaces on for loops
        for (int i = 0; i < 10, i++) {
            // inside of for loops are also indented
            p++;
            q += 3;
            if (q >= 20) // single-line ifs like this are permitted
                System.out.println("bleh");
        }
        
        /*
         * Inner variables don't have to be capitalized properly.
         * 
         * Anonymous classes have a space after parentheses and opening curly
         * brace. The inside is indented.
         */
        SimFunction mysim = new SimFunction() {
            public TeamObj doSim(TeamObj team1, TeamObj team2) {
                return team1;
            }
        };

        return 1037; // parentheses for return exprs aren't necessary
    }
    
    /**
     * Since toString() is already in object, this is technically unnecessary
     */
    @Override
    public String toString() {
        return "style sheet";
    }
    
    /*
     * Order of fields is just like order of methods. "final" goes after
     * non-final but inside the same bracket above. So, "protected final" goes
     * after "public final" or "protected" but before "private".
     * 
     * protected x
     * public static final y
     */
    public int use;
    
    public final String bleh = "bleh";
    
    private float[] arrayExample;
    // variables/fields are named like methods

}
\end{verbatim}
\end{singlespacing}

\pagebreak

\section{Timeline \& Logs}\label{TimelineLogs}
\subsection{Timeline of BALL Development}

\begin{center}
\begin{longtable}{|c|p{10cm}|}
\hline
Time & Event\\
\hline
9/18/09, 11:31a & Team Llamamelon Born.\\
9/21/09, 5:15p & First Team Llamamelon Meeting; Team Llamamelon
named. Sam pioneers the language purpose: the simulation of baseball
games.\\
9/21/09, 8:02p & Team Llamamelon announced to PLT Teaching Staff.\\
10/2/09, 4:42p & Alfred V. Aho tentatively approves BALL programming
language.\\
10/5/09, 1:04a & Buzzwords for initial Whitepaper finalized.\\
10/7/07, 3:17p & Final Whitepaper completed.\\
10/23/09, 2:52p & Sample programs designed, BALL envisioned to be a
scripting language à la AWK.\\
10/27/09, 4:14p & Google Code Group Created. Jordan initializes
Subversion repository and trains team for first 'commit.'\\
10/27/09, 5:30p & First major Team Llamamelon Meeting; language
roadmap created.\\
10/29/09, 10:30a & First meeting with Sinan; BALL changes from
scripted language to compiled language, based on Java.\\
10/30/09, 1:27p & Nathan creates first example program; the new
language roadmap is born.\\
10/31/09, 10:42p & Cipta begins creating the grammar, laying the
groundwork for BALL.\\
11/3/09, 1:05p & Language Reference Manual and Tutorial begun.\\
11/3/09, 4:43p & Language Grammar completed, Daniel creates and
finalizes complex simulation to show off the powerful grammar.\\
11/4/09, 10:26a & Final Language Reference Manual and Tutorial
completed.\\
11/15/09, 3:00p & First semi-weekly meeting. Full grammar implemented
using JFlex/BYACC-J. Lexer created a DFA!\\
11/18/09, 8:00p & Second semi-weekly meeting. Play.ball designed,
intermediate code generation begins.\\
11/19/09, 11:00a & Second meeting with Sinan; Language Reference
Manual and Tutorial approved; BALL is in full motion!\\
11/27/09, 12:43a & Play.Ball compiles and runs: ``Play Ball!'' [Hello
  world equivalent.]\\
12/3/09, 3:45p & Team Loader created; teams can now load into the
PlayerObj and TeamObj constructs.\\
12/5/09, 11:46a & Major all-day meeting; BALL presentation created and
beautified.\\
12/7/09, 4:30p & BALL presented to PLT. There was much rejoicing.\\
12/10/09, 4:37a & SymbolTable embraced as major component of BALL.\\
12/10/09, 3:02a & Simfunc.ball compiles and runs: simfunctions and
activate work.\\
12/12/09, 10:00a & Major all-day meeting; Much of language completed
during this time.\\
12/12/09, 10:48a & TeamStat and PlayerStat methods added: stats work
throughout BALL.\\
12/12/09, 7:59p & Lists become typed, major change in the language.\\
12/12/09, 8:55p & Logical, Boolean, Comparison expressions completed.\\
12/15/09, 12:56a & \texttt{where}, \texttt{from}, and \texttt{any} keywords working.\\
12/16/09, 2:02a & \texttt{'s} keyword works for all attributes, stats, and
lists inherent in team and player objects.\\
12/16/09, 6:08p & Unary operators completed.\\
12/17/09, 6:27p & Realized the class number is W4115, not
W4119. [Seriously.]\\
12/20/09, 10:30a & Final Major all-day meeting: everything completed.\\
12/20/09, 8:04p & Everything in the language works. Final presentation
compilation begun.\\
Now & Final presentation completed.\\

\hline
\caption{Token Example}
\label{token example}
\end{longtable}
\end{center}

\subsection{Project Log}

\begin{center}
\begin{supertabular}{|c|c|c|c|}

\hline
r540 & daniellasry & 2009-12-20 04:58:48 -0500 (Sun, 20 Dec 2009) & 2 lines\\
\multicolumn{4}{|l|}{cleaned up unit testing} \\
\hline
r539 & jordan.schau & 2009-12-20 04:43:57 -0500 (Sun, 20 Dec 2009) & 1 line \\
\multicolumn{4}{|p{8cm}|}{its late and i forgot 2 ;'s.  its late} \\
\hline
r538 & jordan.schau & 2009-12-20 04:43:18 -0500 (Sun, 20 Dec 2009) & 1 line \\
\multicolumn{4}{|p{8cm}|}{regression test = done.  all hundo lines or so} \\
\hline
r537 & daniellasry & 2009-12-20 04:42:22 -0500 (Sun, 20 Dec 2009) & 3 lines \\
\multicolumn{4}{|p{8cm}|}{TO CHECK FOR NULL TYPE} \\
\multicolumn{4}{|p{8cm}|}{-Use the function isValid() :)} \\
\hline
r536 & jordan.schau & 2009-12-20 04:32:30 -0500 (Sun, 20 Dec 2009) & 3 lines \\
\multicolumn{4}{|p{8cm}|}{DONT FORGET THE END!!!! ALWAYS CHECK FOR THE END!!!!!!!!   } \\
\multicolumn{4}{|p{8cm}|}{cue "The End" by the doors.  Seems appropriate eh?} \\
\hline
r535 & jordan.schau & 2009-12-20 04:15:20 -0500 (Sun, 20 Dec 2009) & 1 line \\
\multicolumn{4}{|p{8cm}|}{broken test!  (as it should!)} \\
\hline
r534 & daniellasry & 2009-12-20 03:30:12 -0500 (Sun, 20 Dec 2009) & 2 lines \\
\multicolumn{4}{|p{8cm}|}{Fixed initializing empty lists} \\
\hline
r533 & jordan.schau & 2009-12-20 03:29:05 -0500 (Sun, 20 Dec 2009) & 1 line \\
\multicolumn{4}{|p{8cm}|}{Testing suite and bug fixes} \\
\hline
r532 & daniellasry & 2009-12-20 02:03:37 -0500 (Sun, 20 Dec 2009) & 2 lines \\
\multicolumn{4}{|p{8cm}|}{More error reporting} \\
\hline
r531 & daniellasry & 2009-12-20 01:41:48 -0500 (Sun, 20 Dec 2009) & 2 lines \\
\multicolumn{4}{|p{8cm}|}{Committing latest changes in error reporting} \\
\hline
r530 & jordan.schau & 2009-12-20 01:24:49 -0500 (Sun, 20 Dec 2009) & 1 line \\
\multicolumn{4}{|p{8cm}|}{Commitment.} \\
\hline
r529 & jordan.schau & 2009-12-20 01:24:37 -0500 (Sun, 20 Dec 2009) & 1 line \\
\multicolumn{4}{|p{8cm}|}{Almost done with unit testing.  fiiiiinally} \\
\hline
r528 & jordan.schau & 2009-12-20 01:14:25 -0500 (Sun, 20 Dec 2009) & 6 lines \\
\multicolumn{4}{|p{8cm}|}{Captains Log - Day 43} \\
\multicolumn{4}{|p{8cm}|}{Unit testing has been completed} \\
\multicolumn{4}{|p{8cm}|}{The water is running out and the men have scurvy.} \\
\multicolumn{4}{|p{8cm}|}{Yet, I am not too worried.  } \\
\multicolumn{4}{|p{8cm}|}{Game on.} \\
\hline
r527 & jordan.schau & 2009-12-20 01:05:44 -0500 (Sun, 20 Dec 2009) & 1 line \\
\multicolumn{4}{|p{8cm}|}{more tests complete- stay tuned for more} \\
\hline
r526 & jordan.schau & 2009-12-20 00:46:59 -0500 (Sun, 20 Dec 2009) & 1 line \\
\multicolumn{4}{|p{8cm}|}{Test complete.~} \\
\hline
r525 & jordan.schau & 2009-12-20 00:39:12 -0500 (Sun, 20 Dec 2009) & 1 line \\
\multicolumn{4}{|p{8cm}|}{TEST COMPLETE! (ROBOT VOICE)} \\
\hline
r524 & jordan.schau & 2009-12-20 00:29:30 -0500 (Sun, 20 Dec 2009) & 1 line \\
\multicolumn{4}{|p{8cm}|}{Testing - SQUASHED BUG THAT WAS TEAMNAME = DEATH!  (NOT KIDDING)} \\
\hline
r523 & slee82 & 2009-12-20 00:22:07 -0500 (Sun, 20 Dec 2009) & 1 line \\
\hline
r522 & nathanmayermiller & 2009-12-20 00:08:09 -0500 (Sun, 20 Dec 2009) & 1 line \\
\multicolumn{4}{|p{8cm}|}{Added four tester teams: Amazin' Mavens, Awful Waffles, Hittin' Kittens, Pitchin' Bitches} \\
\hline
r521 & slee82 & 2009-12-20 00:03:30 -0500 (Sun, 20 Dec 2009) & 1 line \\
\multicolumn{4}{|p{8cm}|}{BALL shell script rev2} \\
\hline
r520 & nathanmayermiller & 2009-12-19 23:50:15 -0500 (Sat, 19 Dec 2009) & 1 line \\
\multicolumn{4}{|p{8cm}|}{Fixed Astros, added Cards and Cubs} \\
\hline
r519 & jordan.schau & 2009-12-19 23:45:08 -0500 (Sat, 19 Dec 2009) & 1 line \\
\multicolumn{4}{|p{8cm}|}{American League EAST~~ BOOM BOOM } \\
\hline
r518 & jordan.schau & 2009-12-19 23:39:09 -0500 (Sat, 19 Dec 2009) & 1 line \\
\multicolumn{4}{|p{8cm}|}{OOOOOOH CAAAAAAA~~NAAAAA~~DAAAAAAAAAAAAAAAAAAAAAAAAAAAAAAA!!!!!!!} \\
\hline
r517 & jordan.schau & 2009-12-19 23:35:56 -0500 (Sat, 19 Dec 2009) & 1 line \\
\multicolumn{4}{|p{8cm}|}{rays (not the devil rays)} \\
\hline
r516 & jordan.schau & 2009-12-19 23:33:07 -0500 (Sat, 19 Dec 2009) & 1 line \\
\multicolumn{4}{|p{8cm}|}{red sox (socks?)} \\
\hline
r515 & slee82 & 2009-12-19 23:31:40 -0500 (Sat, 19 Dec 2009) & 1 line \\
\multicolumn{4}{|p{8cm}|}{BALL sh script... rev1 takes cmd line argument the .ball source file} \\
\hline
r514 & jordan.schau & 2009-12-19 23:29:55 -0500 (Sat, 19 Dec 2009) & 1 line \\
\multicolumn{4}{|p{8cm}|}{yankees \#1  (if you dont count the dodgers!)} \\
\hline
r513 & slee82 & 2009-12-19 23:22:29 -0500 (Sat, 19 Dec 2009) & 1 line \\
\multicolumn{4}{|p{8cm}|}{updated ball.lex \& ball.y, removed system.err messages}\\
\hline
r512 & jordan.schau & 2009-12-19 23:21:28 -0500 (Sat, 19 Dec 2009) & 1 line \\
\multicolumn{4}{|p{8cm}|}{dbacks hello!} \\
\hline
r511 & jordan.schau & 2009-12-19 23:17:01 -0500 (Sat, 19 Dec 2009) & 1 line \\
\multicolumn{4}{|p{8cm}|}{padres - vamos los doyers!} \\
\hline
r510 & nathanmayermiller & 2009-12-19 23:15:42 -0500 (Sat, 19 Dec 2009) & 1 line \\
\multicolumn{4}{|p{8cm}|}{Added: Braves, Marlins, Mets, Nats, Phils. BOOM. NL EAST COMPLETE.} \\
\hline
r509 & jordan.schau & 2009-12-19 23:10:55 -0500 (Sat, 19 Dec 2009) & 1 line \\
\multicolumn{4}{|p{8cm}|}{adding giants - more teams MORE FUN!} \\
\hline
r508 & daniellasry & 2009-12-19 23:04:34 -0500 (Sat, 19 Dec 2009) & 2 lines \\
\multicolumn{4}{|p{8cm}|}{EVERYTHING FINALLY WORKS} \\
\hline
r507 & ciphwn & 2009-12-19 22:43:14 -0500 (Sat, 19 Dec 2009) & 1 line \\
\multicolumn{4}{|p{8cm}|}{simnectar working (hopefully)} \\
\hline
r505 & ciphwn & 2009-12-19 22:15:11 -0500 (Sat, 19 Dec 2009) & 1 line \\
\multicolumn{4}{|p{8cm}|}{working simnectar} \\
\hline
r502 & daniellasry & 2009-12-19 21:48:48 -0500 (Sat, 19 Dec 2009) & 3 lines \\
\multicolumn{4}{|p{8cm}|}{Correct simNectar} \\
\hline
r501 & daniellasry & 2009-12-19 21:25:29 -0500 (Sat, 19 Dec 2009) & 5 lines \\
\multicolumn{4}{|p{8cm}|}{MERGING TO TRUNK! committing :} \\
\multicolumn{4}{|p{8cm}|}{- HAPPY EASTER} \\
\multicolumn{4}{|p{8cm}|}{- error reporting} \\
\hline
r499 & nathanmayermiller & 2009-12-19 19:57:26 -0500 (Sat, 19 Dec 2009) & 1 line \\
\multicolumn{4}{|p{8cm}|}{Changed team files.} \\
\hline
r498 & nathanmayermiller & 2009-12-19 19:52:57 -0500 (Sat, 19 Dec 2009) & 1 line \\
\multicolumn{4}{|p{8cm}|}{Added HR stat to pitchers. NOICE.} \\
\hline
r497 & ciphwn & 2009-12-19 19:47:51 -0500 (Sat, 19 Dec 2009) & 3 lines \\
\multicolumn{4}{|p{8cm}|}{1. Iteration stmt counting uses auto-generated ID} \\
\multicolumn{4}{|p{8cm}|}{2. added bottom... functions} \\
\multicolumn{4}{|p{8cm}|}{3. statlist getting returns list of PlayerObj not list of Object} \\
\hline
r496 & nathanmayermiller & 2009-12-19 19:25:13 -0500 (Sat, 19 Dec 2009) & 1 line \\
\multicolumn{4}{|p{8cm}|}{Fixed ApostrExpr for teamname attribute.} \\
\hline
r494 & nathanmayermiller & 2009-12-19 19:17:28 -0500 (Sat, 19 Dec 2009) & 1 line \\
\multicolumn{4}{|p{8cm}|}{Added teamname attribute.} \\
\hline
r493 & ciphwn & 2009-12-19 19:09:49 -0500 (Sat, 19 Dec 2009) & 1 line \\
\multicolumn{4}{|p{8cm}|}{comitting simNectar.ball, compiles in BALL but javac fails} \\
\hline
r492 & ciphwn & 2009-12-19 18:52:40 -0500 (Sat, 19 Dec 2009) & 1 line \\
\multicolumn{4}{|p{8cm}|}{fixed foreach so identifier doesn't have to be declared, but unique} \\
\hline
r491 & ciphwn & 2009-12-19 18:31:26 -0500 (Sat, 19 Dec 2009) & 2 lines \\
\multicolumn{4}{|p{8cm}|}{1. Adding Sam's loop statements code} \\
\multicolumn{4}{|p{8cm}|}{2. "self" -> "item"} \\
\hline
r490 & ciphwn & 2009-12-19 18:12:39 -0500 (Sat, 19 Dec 2009) & 2 lines \\
\multicolumn{4}{|p{8cm}|}{1. fixed negative numbers with unary minus} \\
\multicolumn{4}{|p{8cm}|}{2. prevented deletion of files with '-k' in Run.java} \\
\hline
r483 & ciphwn & 2009-12-19 17:36:13 -0500 (Sat, 19 Dec 2009) & 1 line \\
\multicolumn{4}{|p{8cm}|}{deleting the useless 'testing' package.} \\
\hline
r482 & ciphwn & 2009-12-19 17:29:13 -0500 (Sat, 19 Dec 2009) & 1 line \\
\multicolumn{4}{|p{8cm}|}{Implemented topPlayers, topTeams} \\
\hline
r477 & slee82 & 2009-12-19 15:08:34 -0500 (Sat, 19 Dec 2009) & 1 line \\
\multicolumn{4}{|p{8cm}|}{merged if stmts and loop stmts} \\
\hline
r476 & ciphwn & 2009-12-19 14:49:50 -0500 (Sat, 19 Dec 2009) & 1 line \\
\multicolumn{4}{|p{8cm}|}{added builtins for top, etc} \\
\hline
r475 & slee82 & 2009-12-19 14:41:14 -0500 (Sat, 19 Dec 2009) & 1 line \\
\multicolumn{4}{|p{8cm}|}{s.ball} \\
\hline
r474 & nathanmayermiller & 2009-12-19 13:00:31 -0500 (Sat, 19 Dec 2009) & 1 line \\
\multicolumn{4}{|p{8cm}|}{Fixed fixFloat} \\
\hline
r473 & nathanmayermiller & 2009-12-19 12:33:55 -0500 (Sat, 19 Dec 2009) & 1 line \\
\multicolumn{4}{|p{8cm}|}{Added BF to Program.java. Chicka Chicka Boom Boom - UNN} \\
\hline
r472 & slee82 & 2009-12-19 12:33:01 -0500 (Sat, 19 Dec 2009) & 1 line \\
\multicolumn{4}{|p{8cm}|}{fixFloat to PrintStmt.java} \\
\hline
r471 & slee82 & 2009-12-19 12:31:53 -0500 (Sat, 19 Dec 2009) & 1 line \\
\multicolumn{4}{|p{8cm}|}{fixFloat to string concats in ArithmeticExpr.java} \\
\hline
r470 & nathanmayermiller & 2009-12-19 12:31:39 -0500 (Sat, 19 Dec 2009) & 1 line \\
\multicolumn{4}{|p{8cm}|}{Fixed PlayerObj BF assignment/constructor problem. -UNN} \\
\hline
r469 & nathanmayermiller & 2009-12-19 12:30:13 -0500 (Sat, 19 Dec 2009) & 1 line \\
\multicolumn{4}{|p{8cm}|}{Fixed parenthesis.} \\
\hline
r468 & slee82 & 2009-12-19 12:29:03 -0500 (Sat, 19 Dec 2009) & 1 line \\
\multicolumn{4}{|p{8cm}|}{added fixFloat to Tools.java} \\
\hline
r467 & nathanmayermiller & 2009-12-19 12:28:07 -0500 (Sat, 19 Dec 2009) & 1 line \\
\multicolumn{4}{|p{8cm}|}{Team files updated to add BF stat. BOOYAKASHA} \\
\hline
r466 & slee82 & 2009-12-19 12:28:04 -0500 (Sat, 19 Dec 2009) & 1 line \\
\multicolumn{4}{|p{8cm}|}{added BF to Loader.java} \\
\hline
r465 & nathanmayermiller & 2009-12-19 12:22:16 -0500 (Sat, 19 Dec 2009) & 1 line \\
\multicolumn{4}{|p{8cm}|}{Batters faced tester.} \\
\hline
r464 & slee82 & 2009-12-19 12:17:41 -0500 (Sat, 19 Dec 2009) & 1 line \\
\multicolumn{4}{|p{8cm}|}{added BF to pitchers in Loader.java } \\
\hline
r463 & nathanmayermiller & 2009-12-19 12:15:23 -0500 (Sat, 19 Dec 2009) & 1 line \\
\multicolumn{4}{|p{8cm}|}{Fixed Program.java to have 2B and 3B instead of DBL and TPL.} \\
\hline
r462 & nathanmayermiller & 2009-12-19 12:13:06 -0500 (Sat, 19 Dec 2009) & 1 line \\
\multicolumn{4}{|p{8cm}|}{Semi-colon error. OOP! Goofy me!} \\
\hline
r461 & nathanmayermiller & 2009-12-19 12:10:34 -0500 (Sat, 19 Dec 2009) & 1 line \\
\multicolumn{4}{|p{8cm}|}{Added BF methods/attribute to PlayerObj.} \\
\hline
r458 & daniellasry & 2009-12-19 01:52:21 -0500 (Sat, 19 Dec 2009) & 5 lines \\
\multicolumn{4}{|p{8cm}|}{IMPORVED ERROR REPORTING!} \\
\multicolumn{4}{|p{8cm}|}{- The parser now tells you the line number and the last matched token!} \\
\multicolumn{4}{|p{8cm}|}{- type 'prit "hello"' and see for yourselves} \\
\hline
r457 & daniellasry & 2009-12-19 00:34:25 -0500 (Sat, 19 Dec 2009) & 2 lines \\
\multicolumn{4}{|p{8cm}|}{MERGED!} \\
\hline
r429 & jordan.schau & 2009-12-16 21:08:34 -0500 (Wed, 16 Dec 2009) & 1 line \\
\hline
r423 & ciphwn & 2009-12-16 05:02:13 -0500 (Wed, 16 Dec 2009) & 3 lines \\
\multicolumn{4}{|p{8cm}|}{Merging nathan\_1 to trunk.} \\
\multicolumn{4}{|p{8cm}|}{changes: 's working} \\
\hline
r414 & ciphwn & 2009-12-15 17:14:36 -0500 (Tue, 15 Dec 2009) & 2 lines \\
\multicolumn{4}{|p{8cm}|}{1. fixed auto-runner to delete children class files} \\
\multicolumn{4}{|p{8cm}|}{2. stat semicolon/access problem fixd} \\
\hline
r412 & ciphwn & 2009-12-15 14:17:16 -0500 (Tue, 15 Dec 2009) & 1 line \\
\multicolumn{4}{|p{8cm}|}{Added auto compiler/runner. needs tools.jar.} \\
\hline
r410 & ciphwn & 2009-12-15 04:27:01 -0500 (Tue, 15 Dec 2009) & 4 lines \\
\multicolumn{4}{|p{8cm}|}{Merged cipta\_7 to trunk.} \\
\multicolumn{4}{|p{8cm}|}{1. Style sheet} \\
\multicolumn{4}{|p{8cm}|}{2. 'any' and 'from'} \\
\multicolumn{4}{|p{8cm}|}{3. Tools.java where random, top is probably going to be put} \\
\hline
r402 & ciphwn & 2009-12-14 18:59:27 -0500 (Mon, 14 Dec 2009) & 8 lines \\
\multicolumn{4}{|p{8cm}|}{COMMITING CIPTA\_6 TO TRUNK} \\
\multicolumn{4}{|p{8cm}|}{CHANGES} \\
\multicolumn{4}{|p{8cm}|}{1. 'where' is working. To get 'where' working expressions can now insert statements before the statement it currently lives in to pre-evaluate stuff (this makes writing loops tricky, but doable)} \\
\multicolumn{4}{|p{8cm}|}{2. Symbol table makes random tokens differently now. Everything starts with "tok\_" and a random  hex number.} \\
\multicolumn{4}{|p{8cm}|}{3. statements must check whether they have accumulated any inserts because of the expressions inside them. This is done in the "Stmt" base class. Subclasses need not worry but they must implement the "stmtCode" function instead of "code" now.} \\
\hline
r388 & jordan.schau & 2009-12-12 21:19:44 -0500 (Sat, 12 Dec 2009) & 1 line \\
\multicolumn{4}{|p{8cm}|}{updated to include the bool in Type.java and LogicalExpr.java (but shhh bools are a secret to the end user)} \\
\hline
r387 & ciphwn & 2009-12-12 20:15:00 -0500 (Sat, 12 Dec 2009) & 1 line \\
\multicolumn{4}{|p{8cm}|}{merged changes from cipta\_6 to trunk} \\
\hline
r382 & ciphwn & 2009-12-12 16:43:47 -0500 (Sat, 12 Dec 2009) & 1 line \\
\multicolumn{4}{|p{8cm}|}{Merge list lintializer to trunk} \\
\hline
r379 & jordan.schau & 2009-12-12 16:31:25 -0500 (Sat, 12 Dec 2009) & 1 line \\
\hline
r378 & jordan.schau & 2009-12-12 15:45:45 -0500 (Sat, 12 Dec 2009) & 1 line \\
\multicolumn{4}{|p{8cm}|}{arithmetic expressions are working - butter} \\
\hline
r366 & ciphwn & 2009-12-12 13:52:05 -0500 (Sat, 12 Dec 2009) & 1 line \\
\multicolumn{4}{|p{8cm}|}{myfun2 syntax error} \\
\hline
r364 & nathanmayermiller & 2009-12-12 13:48:38 -0500 (Sat, 12 Dec 2009) & 1 line \\
\multicolumn{4}{|p{8cm}|}{PlayerStat methods; changed double ip to float ip.} \\
\hline
r363 & ciphwn & 2009-12-12 13:45:13 -0500 (Sat, 12 Dec 2009) & 1 line \\
\multicolumn{4}{|p{8cm}|}{added '()'} \\
\hline
r362 & nathanmayermiller & 2009-12-12 13:41:10 -0500 (Sat, 12 Dec 2009) & 1 line \\
\multicolumn{4}{|p{8cm}|}{Added TeamStat methods} \\
\hline
r361 & ciphwn & 2009-12-12 13:30:42 -0500 (Sat, 12 Dec 2009) & 4 lines \\
\multicolumn{4}{|p{8cm}|}{- merged stats with simfunction} \\
\multicolumn{4}{|p{8cm}|}{- added smart code indenting} \\
\multicolumn{4}{|p{8cm}|}{- a few builtin stats} \\
\multicolumn{4}{|p{8cm}|}{- original expr => AtomicExpr} \\
\hline
r355 & daniellasry & 2009-12-10 18:03:32 -0500 (Thu, 10 Dec 2009) & 2 lines \\
\multicolumn{4}{|p{8cm}|}{Committed my branch to the trunk. Check out simfunc.ball} \\
\hline
r341 & daniellasry & 2009-12-09 20:13:06 -0500 (Wed, 09 Dec 2009) & 2 lines \\
\multicolumn{4}{|p{8cm}|}{Tested sim() a little more, going to brach off soon and implement activation statements.} \\
\hline
r340 & daniellasry & 2009-12-09 19:45:23 -0500 (Wed, 09 Dec 2009) & 2 lines \\
\multicolumn{4}{|p{8cm}|}{sim() working} \\
\hline
r339 & daniellasry & 2009-12-09 19:05:23 -0500 (Wed, 09 Dec 2009) & 2 lines \\
\multicolumn{4}{|p{8cm}|}{Modified javabackend: added a sim function, fixed IP in playerObj} \\
\hline
r330 & ciphwn & 2009-12-09 17:12:07 -0500 (Wed, 09 Dec 2009) & 2 lines \\
\multicolumn{4}{|p{8cm}|}{1. Fixed shift/reduce conflict because of empty function body} \\
\multicolumn{4}{|p{8cm}|}{2. Added special declaration for functions with empty bodies} \\
\hline
r329 & daniellasry & 2009-12-09 17:08:46 -0500 (Wed, 09 Dec 2009) & 2 lines \\
\multicolumn{4}{|p{8cm}|}{added regexp decimal numbers} \\
\hline
r327 & daniellasry & 2009-12-09 15:45:16 -0500 (Wed, 09 Dec 2009) & 2 lines \\
\multicolumn{4}{|p{8cm}|}{Working on the'sim()' function} \\
\hline
r323 & daniellasry & 2009-12-09 05:56:11 -0500 (Wed, 09 Dec 2009) & 2 lines \\
\multicolumn{4}{|p{8cm}|}{Removing presentations to save space (it adds up every time we branch) I will add it to the downloads.} \\
\hline
r322 & daniellasry & 2009-12-09 05:52:18 -0500 (Wed, 09 Dec 2009) & 2 lines \\
\multicolumn{4}{|p{8cm}|}{Going to bed: MERGING WITH TRUNK} \\
\hline
r318 & ciphwn & 2009-12-09 02:31:11 -0500 (Wed, 09 Dec 2009) & 10 lines \\
\multicolumn{4}{|p{8cm}|}{Functions and variable declarations now completely follow how the ball source is declared. That is, this is illegal:} \\
\multicolumn{4}{|p{8cm}|}{function x() returns nothing:} \\
\multicolumn{4}{|p{8cm}|}{	y();} \\
\multicolumn{4}{|p{8cm}|}{end} \\
\multicolumn{4}{|p{8cm}|}{function y() returns nothing:} \\
\multicolumn{4}{|p{8cm}|}{end} \\
\multicolumn{4}{|p{8cm}|}{because x doesn't know of y yet.} \\
\hline
r317 & ciphwn & 2009-12-09 01:56:19 -0500 (Wed, 09 Dec 2009) & 6 lines \\
\multicolumn{4}{|p{8cm}|}{Commiting changes from cipta\_3 branch to main trunk.} \\
\multicolumn{4}{|p{8cm}|}{1. Function definition basics working} \\
\multicolumn{4}{|p{8cm}|}{2. Variable declarations, both in the top level and inside function blocks, work. Top level variables gets pushed outside the class, vars inside functions stay inside.} \\
\multicolumn{4}{|p{8cm}|}{3. Assignment basics work. Variables can now be redefined.} \\
\multicolumn{4}{|p{8cm}|}{4. Type checking basics. Since there are only a handful of types this isn't hard to do.} \\
\hline
r304 & daniellasry & 2009-12-07 15:30:20 -0500 (Mon, 07 Dec 2009) & 2 lines \\
\multicolumn{4}{|p{8cm}|}{Fixed something} \\
\hline
r303 & daniellasry & 2009-12-07 15:28:08 -0500 (Mon, 07 Dec 2009) & 2 lines \\
\multicolumn{4}{|p{8cm}|}{Variable declarations now happen outside 'main' (on top of main).} \\
\hline
r299 & daniellasry & 2009-12-07 13:27:20 -0500 (Mon, 07 Dec 2009) & 2 lines \\
\multicolumn{4}{|p{8cm}|}{commited pres} \\
\hline
r298 & daniellasry & 2009-12-05 20:40:23 -0500 (Sat, 05 Dec 2009) & 2 lines \\
\multicolumn{4}{|p{8cm}|}{Updated presentation} \\
\hline
r297 & daniellasry & 2009-12-05 17:28:05 -0500 (Sat, 05 Dec 2009) & 3 lines \\
\multicolumn{4}{|p{8cm}|}{updated presentation} \\
\hline
r296 & nathanmayermiller & 2009-12-05 15:39:10 -0500 (Sat, 05 Dec 2009) & 1 line \\
\multicolumn{4}{|p{8cm}|}{Added comments to play.ball to test it.} \\
\hline
r295 & nathanmayermiller & 2009-12-05 12:57:41 -0500 (Sat, 05 Dec 2009) & 1 line \\
\multicolumn{4}{|p{8cm}|}{Fixed IP values (.1,.2) for playerObj} \\
\hline
r294 & daniellasry & 2009-12-05 12:22:47 -0500 (Sat, 05 Dec 2009) & 2 lines \\
\multicolumn{4}{|p{8cm}|}{Added The presentation} \\
\hline
r293 & daniellasry & 2009-12-04 17:15:05 -0500 (Fri, 04 Dec 2009) & 2 lines \\
\multicolumn{4}{|p{8cm}|}{Added support for integers. I can't get the regexp for decimals to work yet.} \\
\hline
r292 & jordan.schau & 2009-12-04 11:20:06 -0500 (Fri, 04 Dec 2009) & 3 lines \\
\multicolumn{4}{|p{8cm}|}{adding Daniels load application from the email he sent out to test it (as the tester and validator - because thats my job)} \\
\hline
r291 & daniellasry & 2009-12-04 04:30:42 -0500 (Fri, 04 Dec 2009) & 2 lines \\
\multicolumn{4}{|p{8cm}|}{Goign to add support for numbers tomorrow} \\
\hline
r290 & daniellasry & 2009-12-04 04:02:36 -0500 (Fri, 04 Dec 2009) & 2 lines \\
\multicolumn{4}{|p{8cm}|}{In this revision, play.ball reads two teams and prints them!!!!} \\
\hline
r289 & daniellasry & 2009-12-04 02:29:22 -0500 (Fri, 04 Dec 2009) & 2 lines \\
\multicolumn{4}{|p{8cm}|}{cleaned up more} \\
\hline
r288 & daniellasry & 2009-12-04 02:26:30 -0500 (Fri, 04 Dec 2009) & 2 lines \\
\multicolumn{4}{|p{8cm}|}{cleaned junk} \\
\hline
r287 & daniellasry & 2009-12-04 02:24:16 -0500 (Fri, 04 Dec 2009) & 2 lines \\
\multicolumn{4}{|p{8cm}|}{Merged my branch into trunk} \\
\hline
r283 & daniellasry & 2009-12-03 15:11:33 -0500 (Thu, 03 Dec 2009) & 2 lines \\
\multicolumn{4}{|p{8cm}|}{Added the Loader class. Worth a look. Returns a team. Handles a lot of errors.} \\
\hline
r281 & jordan.schau & 2009-12-03 14:22:07 -0500 (Thu, 03 Dec 2009) & 1 line \\
\multicolumn{4}{|p{8cm}|}{trivial comment change to test eclipse commit } \\
\hline
r280 & jordan.schau & 2009-12-03 13:59:13 -0500 (Thu, 03 Dec 2009) & 2 lines \\
\multicolumn{4}{|p{8cm}|}{updated astros and dodgers to match the data on baseball-reference.  removed the .txt file extension} \\
\hline
r279 & jordan.schau & 2009-12-03 13:30:09 -0500 (Thu, 03 Dec 2009) & 4 lines \\
\multicolumn{4}{|p{8cm}|}{Added ref man and tut.  moved to docs folder.} \\
\multicolumn{4}{|p{8cm}|}{xo} \\
\hline
r278 & daniellasry & 2009-12-03 08:03:58 -0500 (Thu, 03 Dec 2009) & 2 lines \\
\multicolumn{4}{|p{8cm}|}{Realized that Innings Pitches are actually decimal numbers, not ints. Reverted to the old CSVs and modified the player class to accept doubles for innings pitched.} \\
\hline
r277 & daniellasry & 2009-12-03 07:55:49 -0500 (Thu, 03 Dec 2009) & 2 lines \\
\multicolumn{4}{|p{8cm}|}{Added toString methods to the teams and Players. The load function works, commenting it.} \\
\hline
r276 & daniellasry & 2009-12-03 07:07:18 -0500 (Thu, 03 Dec 2009) & 3 lines \\
\multicolumn{4}{|p{8cm}|}{Fixed the constants in playerObj to be public final static int (so they can be accessed and yet not modified).} \\
\multicolumn{4}{|p{8cm}|}{Removed some random decimal values in the innings pitched (???).} \\
\hline
r275 & daniellasry & 2009-12-03 02:28:05 -0500 (Thu, 03 Dec 2009) & 2 lines \\
\multicolumn{4}{|p{8cm}|}{Removed unnecessary "Team Stats" header which made parsing difficult. Fixed Expr and PrintStmt so that the java print statement appears on one line.} \\
\hline
r274 & slee82 & 2009-12-02 21:33:12 -0500 (Wed, 02 Dec 2009) & 3 lines \\
\multicolumn{4}{|p{8cm}|}{Holler at cha boy. What up. TeamObj and playerObj are done. Uppercase Team, Lowercase player. SORRY, I messed up.} \\
\hline
r273 & jordan.schau & 2009-12-02 21:12:14 -0500 (Wed, 02 Dec 2009) & 2 lines \\
\multicolumn{4}{|p{8cm}|}{added teams!} \\
\hline
r272 & ciphwn & 2009-12-02 20:49:11 -0500 (Wed, 02 Dec 2009) & 1 line \\
\multicolumn{4}{|p{8cm}|}{adding next target program} \\
\hline
r271 & ciphwn & 2009-11-27 00:32:41 -0500 (Fri, 27 Nov 2009) & 5 lines \\
\multicolumn{4}{|p{8cm}|}{\_\_Integrating changes from the cipta\_1 branch\_\_} \\
\multicolumn{4}{|p{8cm}|}{1. Can print a hello world program to stdout (redirect stdout to a file to get the java program)} \\
\multicolumn{4}{|p{8cm}|}{2. Uses packages} \\
\multicolumn{4}{|p{8cm}|}{3. Uses an Ant build file} \\
\hline
r260 & ciphwn & 2009-11-23 14:01:01 -0500 (Mon, 23 Nov 2009) & 4 lines \\
\multicolumn{4}{|p{8cm}|}{Merging changes from branch to trunk.} \\
\multicolumn{4}{|p{8cm}|}{- added more comments} \\
\multicolumn{4}{|p{8cm}|}{- added header} \\
\multicolumn{4}{|p{8cm}|}{- added declaration for a symbol table, though not implemented yet} \\
\hline
r247 & jordan.schau & 2009-11-18 22:17:38 -0500 (Wed, 18 Nov 2009) & 2 lines \\
\multicolumn{4}{|p{8cm}|}{Deleting yacc.macosx} \\
\hline
r246 & jordan.schau & 2009-11-18 21:50:45 -0500 (Wed, 18 Nov 2009) & 2 lines \\
\multicolumn{4}{|p{8cm}|}{removing versions} \\
\hline
r245 & ciphwn & 2009-11-18 21:14:04 -0500 (Wed, 18 Nov 2009) & 4 lines \\
\multicolumn{4}{|p{8cm}|}{Removed 'y.tab.c'} \\
\multicolumn{4}{|p{8cm}|}{added test program.} \\
\hline
r244 & ciphwn & 2009-11-18 21:11:03 -0500 (Wed, 18 Nov 2009) & 3 lines \\
\multicolumn{4}{|p{8cm}|}{- Adding clean target} \\
\multicolumn{4}{|p{8cm}|}{- changed "ball\_simple.y" to just "ball.y"} \\
\hline
r243 & jordan.schau & 2009-11-18 21:07:18 -0500 (Wed, 18 Nov 2009) & 2 lines \\
\multicolumn{4}{|p{8cm}|}{removing junk} \\
\hline
r242 & ciphwn & 2009-11-18 21:06:06 -0500 (Wed, 18 Nov 2009) & 5 lines \\
\multicolumn{4}{|p{8cm}|}{First working lex and yacc file} \\
\multicolumn{4}{|p{8cm}|}{- no syntax directed translation yet} \\
\multicolumn{4}{|p{8cm}|}{- successfully detected the "print" keyword} \\
\multicolumn{4}{|p{8cm}|}{- successfully parsed the "playball!" string} \\
\hline
r241 & jordan.schau & 2009-11-18 21:04:19 -0500 (Wed, 18 Nov 2009) & 2 lines \\
\multicolumn{4}{|p{8cm}|}{removing my test junk} \\
\hline
r240 & jordan.schau & 2009-11-18 21:03:45 -0500 (Wed, 18 Nov 2009) & 2 lines \\
\multicolumn{4}{|p{8cm}|}{test} \\
\hline
r239 & ciphwn & 2009-11-18 20:58:23 -0500 (Wed, 18 Nov 2009) & 2 lines \\
\multicolumn{4}{|p{8cm}|}{adding a makefile} \\
\hline
r238 & jordan.schau & 2009-11-18 20:58:21 -0500 (Wed, 18 Nov 2009) & 1 line \\
\multicolumn{4}{|p{8cm}|}{added secrets.  SH!!!!} \\
\hline
r237 & jordan.schau & 2009-11-18 20:55:04 -0500 (Wed, 18 Nov 2009) & 1 line \\
\multicolumn{4}{|p{8cm}|}{added whitepaper...  fingers crossed} \\
\hline
r236 & slee82 & 2009-11-18 20:41:13 -0500 (Wed, 18 Nov 2009) & 1 line \\
\multicolumn{4}{|p{8cm}|}{test} \\
\hline
r1 & (no author) & 2009-10-27 19:14:40 -0400 (Tue, 27 Oct 2009) & 1 line \\
\multicolumn{4}{|p{8cm}|}<{Initial directory structure.} \\
\hline

\end{supertabular}
\end{center}



\input{lang_evo.tex}\label{evolution}

\chapter{Translator Architecture}\label{architecture}

\section{Architectural Block Diagram}
Figure \ref{blockdiagram} shows the Architectural Block Diagram of the
BALL compiler. This compiler and its structures are further explained
in Section \ref{interfaces} below.

\begin{figure}[htbp]
  \centering
  \includegraphics{blocks_scaled.png}
  \caption{Architectural Block Diagram}
  \label{blockdiagram}
\end{figure}


\section{Interfaces between Compiler Modules}\label{interfaces}
The BALL compiler's modules can be grouped into five major steps, as
enumerated below. Reference Figure \ref{blockdiagram} for a visual
representation of the interface.
\begin{enumerate}
\item First, the lexer traverses the BALL source program to be
  compiled. The lexer gathers all of the tokens, passing them to the
  parser in the same sequence. The lexer acts in a left-to-right
  method.
\item The parser then parses the sequence according to the grammar,
  matching statements. It creates a rightmost derivation, and thus
  BALL is a LR(1) language. As the parsing goes on, and the parser
  finds matches in the grammar, it adds identifiers (variables and
  functions) to the symbol table.
\item The parser outputs a syntax tree containing nodes for each
  action (actions include print statements, assignment statements,
  expressions, etc.) Every statement has a node in the syntax tree.
\item Each node generates its own Java code equivalent, checking the
  symbol table at compilation to ensure conflicts or incorrect
  declarations or calls do not exist within the code.
\item The Java code is compile by the Java compiler into bytecode. The
  Java Virtual Machine then runs this compiled bytecode as well as the
  libraries supplied in the Java backend runtime classes. These
  libraries enable BALL's syntax simplicity.
\end{enumerate}
\section{Architecture Work Distribution}
Ultimately, as was the case for all of BALL's development, work was
divided evenly, and each member contributed massively to the completed
project by being involved in every part of the process. For example,
intermediate code generation from the syntax tree was written for
different language constructs by different members of the
team. Similarly the parser and symbol table were adapted based on
which section of code a particular team member worked on at a
particular time. Often, team members worked together to solve one
particular part of the language. Thus, each module in BALL is truly a
team effort, as no particular module can be traced to one individual
person.

BALL is the ultimate baseball language. Just as baseball is a team
sport, BALL is a team-created language.

\chapter{Development Environment}\label{DevelEnv}

\section{Software Development Environment}

The software development environment that was chosen to create the
BALL compiler was Eclipse, running on Ubuntu distribution of Linux
OS. Ubuntu was chosen as the OS because it is open source and was
easily available to all team members. Eclipse was chosen specifically
because of its seamless integration with both Java and Subversion
revision control system. Most of the other tools used were selected
because the majority of the team members were already familiar with
them. Google Code was specially helpful in bringing everything
together since it made available to us features that were invaluable
at every stage of our compiler design. For the majority of the early
stages of our project, the record of our brainstorming sessions were
kept in a wiki stored on Google Code. This was a very helpful resource
that made it easy to trace the language and feature set
evolution. Likewise, without the ability to store our version
controlled code on Google Code, the development cycle would have been
extremely difficult to coordinate. In summary the most important tools
used are listed in the section below.

Ubuntu Linux
Eclipse
Java
Google Code
Subversion
JFlex
BYACC/J

\section{Implementation Languages and Tools}

Java was used to develop and build our BALL compiler to leverage the
considerable expertise that our team members possessed. It was also
specially helpful that Java is highly portable and enabled our BALL
compiler to work wherever Java runs. Only Java standard libraries were
used in the creation of our compiler, to simplify and further increase
the portability of BALL. Jflex \& BYACC/J the Java flavor version of
Lex and Yacc were used to build our lexer and parser. Finally Ant was
used to write buildfiles to automate the compilation process of our
package, and a bash shell script is used to compile and run BALL
source files.  


\chapter{TestPlan}\label{TestPlan}

As a group we all felt a test driven development was crucial to bug
free code.  Test Driven Development in our case started with one line
simple grammars and simple programs.  We then would introduce small
additions to the grammar and write small programs to act as unit tests
for said additions to the grammar.  The following is a code snippet
from the unit testing program for arithmetic expressions:

\begin{singlespacing}
\begin{verbatim}

/*
 * Arithmetic Expr Test File
 */

number x, y, z;

x = 0;//x = 0
y = x + 1;//y=1
z = x / 10;//z=0

x = (z + 1) * 100; //x=100

y = z % 5 + 1;//y=1

1+2;//should get commented out in javacode
x+y;

function f1 () returns number:
    x = 13;
    return 15;
end

function f2() returns number:
    y = x * 5 % 6;
    return 2 + 4;
end

print f1() + 3;// 18
print ``f2() = `` + f2();//6
print ``x = `` + x;//13 BECAUSE ITS GLOBAL (WORLDLY)
print ``y = `` + y;//5

\end{verbatim}
\end{singlespacing}

\input{testplan.tex}

\input{conclusion.tex}

%% <== End of hints
%%%%%%%%%%%%%%%%%%%%%%%%%%%%%%%%%%%%%%%%%%%%%%%%%%%%%%%%%%%%%



%%%%%%%%%%%%%%%%%%%%%%%%%%%%%%%%%%%%%%%%%%%%%%%%%%%%%%%%%%%%%
%% BIBLIOGRAPHY AND OTHER LISTS
%%%%%%%%%%%%%%%%%%%%%%%%%%%%%%%%%%%%%%%%%%%%%%%%%%%%%%%%%%%%%
%% A small distance to the other stuff in the table of contents (toc)
\addtocontents{toc}{\protect\vspace*{\baselineskip}}

%% The Bibliography
%% ==> You need a file 'literature.bib' for this.
%% ==> You need to run BibTeX for this (Project | Properties... | Uses BibTeX)
%\addcontentsline{toc}{chapter}{Bibliography} %'Bibliography' into toc
%\nocite{*} %Even non-cited BibTeX-Entries will be shown.
%\bibliographystyle{alpha} %Style of Bibliography: plain / apalike / amsalpha / ...
%\bibliography{literature} %You need a file 'literature.bib' for this.

%% The List of Figures
\clearpage
\addcontentsline{toc}{chapter}{List of Figures}
\listoffigures

%% The List of Tables
\clearpage
\addcontentsline{toc}{chapter}{List of Tables}
\listoftables

\clearpage
\chapter{Source Code}

%%%%%%
% COMPILER
%%%%%%%%
\section{compiler}

\lstinputlisting[label=ball.lex,caption=ball.lex]{../src/compiler/ball.lex}

\lstinputlisting[label=ball.y,caption=ball.y]{../src/compiler/ball.y}

\lstinputlisting[label=Run.java,caption=Run.java]{../src/compiler/Run.java}

\lstinputlisting[label=SymbolTable.java,caption=SymbolTable.java]{../src/compiler/SymbolTable.java}

%%%%%%
% CODE GENERATOR
%%%%%%%%
\section{Code Generator}

\lstinputlisting[label=ActivateStmt.java,caption=ActivateStmt.java]{../src/codegen/ActivateStmt.java}

\lstinputlisting[label=ApostrExpr.java,caption=ApostrExpr.java]{../src/codegen/ApostrExpr.java}

\lstinputlisting[label=ArithmeticExpr.java,caption=ArithmeticExpr.java]{../src/codegen/ArithmeticExpr.java}

\lstinputlisting[label=AssignmentStmt.java,caption=AssignmentStmt.java]{../src/codegen/AssignmentStmt.java}

\lstinputlisting[label=AtomicExpr.java,caption=AtomicExpr.java]{../src/codegen/AtomicExpr.java}

\lstinputlisting[label=BuiltinAttributeDef.java,caption=BuiltinAttributeDef.java]{../src/codegen/BuiltinAttributeDef.java}

\lstinputlisting[label=BuiltinFuncDef.java,caption=BuiltinFuncDef.java]{../src/codegen/BuiltinFuncDef.java}

\lstinputlisting[label=BuiltinStatDef.java,caption=BuiltinStatDef.java]{../src/codegen/BuiltinStatDef.java}

\lstinputlisting[label=BuiltinStatListDef.java,caption=BuiltinStatListDef.java]{../src/codegen/BuiltinStatListDef.java}

\lstinputlisting[label=ComparisonExpr.java,caption=ComparisonExpr.java]{../src/codegen/ComparisonExpr.java}

\lstinputlisting[label=Declaration.java,caption=Declaration.java]{../src/codegen/Declaration.java}

\lstinputlisting[label=Expr.java,caption=Expr.java]{../src/codegen/Expr.java}

\lstinputlisting[label=ExprStmt.java,caption=ExprStmt.java]{../src/codegen/ExprStmt.java}

\lstinputlisting[label=FilterExpr.java,caption=FilterExpr.java]{../src/codegen/FilterExpr.java}

\lstinputlisting[label=Funcall.java,caption=Funcall.java]{../src/codegen/Funcall.java}

\lstinputlisting[label=FuncDef.java,caption=FuncDef.java]{../src/codegen/FuncDef.java}

\lstinputlisting[label=IfStmt.java,caption=IfStmt.java]{../src/codegen/IfStmt.java}

\lstinputlisting[label=InsertionPoint.java,caption=InsertionPoint.java]{../src/codegen/InsertionPoint.java}

\lstinputlisting[label=ListInit.java,caption=ListInit.java]{../src/codegen/ListInit.java}

\lstinputlisting[label=ListType.java,caption=ListType.java]{../src/codegen/ListType.java}

\lstinputlisting[label=LogicalExpr.java,caption=LogicalExpr.java]{../src/codegen/LogicalExpr.java}

\lstinputlisting[label=MatchExpr.java,caption=MatchExpr.java]{../src/codegen/MatchExpr.java}

\lstinputlisting[label=ParseTreeNode.java,caption=ParseTreeNode.java]{../src/codegen/ParseTreeNode.java}

%\lstinputlisting[label=PlayBall.java,caption=PlayBall.java]{../src/codegen/PlayBall.java}

\lstinputlisting[label=PrintStmt.java,caption=PrintStmt.java]{../src/codegen/PrintStmt.java}

\lstinputlisting[label=Program.java,caption=Program.java]{../src/codegen/Program.java}

\lstinputlisting[label=ReturnStmt.java,caption=ReturnStmt.java]{../src/codegen/ReturnStmt.java}

\lstinputlisting[label=SimFuncDef.java,caption=SimFuncDef.java]{../src/codegen/SimFuncDef.java}

\lstinputlisting[label=StatAtom.java,caption=StatAtom.java]{../src/codegen/StatAtom.java}

\lstinputlisting[label=StatDef.java,caption=StatDef.java]{../src/codegen/StatDef.java}

\lstinputlisting[label=StatExpr.java,caption=StatExpr.java]{../src/codegen/StatExpr.java}

\lstinputlisting[label=StatMult.java,caption=StatMult.java]{../src/codegen/StatMult.java}

\lstinputlisting[label=Stmt.java,caption=Stmt.java]{../src/codegen/Stmt.java}

\lstinputlisting[label=StopdoStmt.java,caption=StopdoStmt.java]{../src/codegen/StopdoStmt.java}

\lstinputlisting[label=UnaryExpr.java,caption=UnaryExpr.java]{../src/codegen/UnaryExpr.java}

%%%%%%
% JAVA BACKEND
%%%%%%%%
\section{Java Backend}

\lstinputlisting[label=Attribute.java,caption=Attribute.java]{../src/javabackend/Attribute.java}

\lstinputlisting[label=BallDataType.java,caption=BallDataType.java]{../src/javabackend/BallDataType.java}

\lstinputlisting[label=BallList.java,caption=BallList.java]{../src/javabackend/BallList.java}

\lstinputlisting[label=Loader.java,caption=Loader.java]{../src/javabackend/Loader.java}

\lstinputlisting[label=PlayerObj.java,caption=PlayerObj.java]{../src/javabackend/PlayerObj.java}

\lstinputlisting[label=PlayerStat.java,caption=PlayerStat.java]{../src/javabackend/PlayerStat.java}

\lstinputlisting[label=SimFunction.java,caption=SimFunction.java]{../src/javabackend/SimFunction.java}

\lstinputlisting[label=Simulator.java,caption=Simulator.java]{../src/javabackend/Simulator.java}

\lstinputlisting[label=TeamAttribute.java,caption=TeamAttribute.java]{../src/javabackend/TeamAttribute.java}

\lstinputlisting[label=TeamList.java,caption=TeamList.java]{../src/javabackend/TeamList.java}

\lstinputlisting[label=TeamObj.java,caption=TeamObj.java]{../src/javabackend/TeamObj.java}

\lstinputlisting[label=TeamStat.java,caption=TeamStat.java]{../src/javabackend/TeamStat.java}

\lstinputlisting[label=Tools.java,caption=Tools.java]{../src/javabackend/Tools.java}


%%%%%%
% LEXER
%%%%%%%%
\section{Java Backend}

\lstinputlisting[label=Identifier.java,caption=Identifier.java]{../src/lexer/Identifier.java}

\lstinputlisting[label=Keyword.java,caption=Keyword.java]{../src/lexer/Keyword.java}

\lstinputlisting[label=NumericConst.java,caption=NumericConst.java]{../src/lexer/NumericConst.java}

\lstinputlisting[label=StringConst.java,caption=StringConst.java]{../src/lexer/StringConst.java}

\lstinputlisting[label=Token.java,caption=Token.java]{../src/lexer/Token.java}

\lstinputlisting[label=Type.java,caption=Type.java]{../src/lexer/Type.java}

%%%%%%
% TEST CODE
%%%%%%%%
\section{TEST CODE}

\lstinputlisting[label=testSuite.ball,caption=testSuite.ball]{../src/testSuite.ball}

\lstinputlisting[label=assign.ball,caption=assign.ball]{../src/unit_testing/assign.ball}

\lstinputlisting[label=bf.ball,caption=bf.ball]{../src/unit_testing/bf.ball}

%\lstinputlisting[label=cplx_fun.ball,caption=cplx_fun.ball]{../src/unit_testing/cplx_fun.ball}

\lstinputlisting[label=from.ball,caption=from.ball]{../src/unit_testing/from.ball}

\lstinputlisting[label=func.ball,caption=func.ball]{../src/unit_testing/func.ball}

\lstinputlisting[label=funcfail.ball,caption=funcfail.ball]{../src/unit_testing/funcfail.ball}

\lstinputlisting[label=global.ball,caption=global.ball]{../src/unit_testing/global.ball}

\lstinputlisting[label=ifstmt.ball,caption=ifstmt.ball]{../src/unit_testing/ifstmt.ball}

\lstinputlisting[label=lists.ball,caption=lists.ball]{../src/unit_testing/lists.ball}

\lstinputlisting[label=load.ball,caption=load.ball]{../src/unit_testing/load.ball}

\lstinputlisting[label=loop.ball,caption=loop.ball]{../src/unit_testing/loop.ball}

\lstinputlisting[label=play.ball,caption=play.ball]{../src/unit_testing/play.ball}

\lstinputlisting[label=s.ball,caption=s.ball]{../src/unit_testing/s.ball}

\lstinputlisting[label=simfunc.ball,caption=simfunc.ball]{../src/unit_testing/simfunc.ball}

\lstinputlisting[label=simNectar.ball,caption=simNectar.ball]{../src/unit_testing/simNectar.ball}

\lstinputlisting[label=stats.ball,caption=stats.ball]{../src/unit_testing/stats.ball}

\lstinputlisting[label=top.ball,caption=top.ball]{../src/unit_testing/top.ball}

\lstinputlisting[label=unary.ball,caption=unary.ball]{../src/unit_testing/unary.ball}

\lstinputlisting[label=where.ball,caption=where.ball]{../src/unit_testing/where.ball}




%%%%%%%%%%%%%%%%%%%%%%%%%%%%%%%%%%%%%%%%%%%%%%%%%%%%%%%%%%%%%
%% APPENDICES
%%%%%%%%%%%%%%%%%%%%%%%%%%%%%%%%%%%%%%%%%%%%%%%%%%%%%%%%%%%%%
\appendix
%% ==> Write your text here or include other files.

%\input{FileName} %You need a file 'FileName.tex' for this.


\end{document}

